\section{What Non-Convex Functions Can We Optimize? (Rong Ge)}

Many machine learning problems require optimizing a non-convex objective. In this talk we identify a class of non-convex functions where all local minima are also globally optimal. For such functions, stochastic gradient descent efficiently converges to the global optimum . Several interesting problems are known to have this property, and we will in particular show matrix completion has no spurious local minimum.

Based on joing works with Furong Huang, Jason Lee, Chi Jin, Tengyu Ma, Yang Yuan

Two results:
\begin{enumerate}
\item
Gradient descent with strict saddles
\item
Matrix completion: all local optima are global
\end{enumerate}•
%strict saddle
%matrix assumption : all local optima are global.

The title is an open problem, a problem I'm hoping to understand better.

Why nonconvex? Many ML problems are nonconvex, such as
\begin{itemize}
\item
find the best clustering
\item
learn the best neural networks
\item
find communities in social networks.
\end{itemize}•
Convex relaxations sometimes work, but are too slow. In some problems (ex. neural networks) we don't have alternatives to nonconvex optimization.

In theory nonconvex problems are NP-hard in worst case. In practice, SGD and tuning works okay; it finds solutions with reasonable quality.

Why? In practice we are not in worst-case scenario; it is tractable for real-life instances. What properties of real-life instances can we use?

What do we know in convex optimization? A convex function satisfies
$$
f\pf{x+y}2\le \fc{f(x)+f(y)}2.
$$
When strongly convex, there is a unique global optimum solution. Find by gradient descent (stochastic, accelerated) or second-order methods---Newton's algorithm (trust region, cubic regularization). They are guaranteed to find the global minimum. The geometry of convexity gives algorithms.

In nonconvex optimization, we hope that we can do something similar. Find clean geometric properties for nonconvex functions that allow efficient solutions.

Relaxations to convexity: Quasiconvexity, Convexity from any point to optimal point. These are not sufficient because they all still have one local/global optimum.

This is not true for many problems because there are many symmetries in the objective function. 
\begin{itemize}
\item
Problem asks for multiple components but the components have no ordering. 
%ex. permute columns
%location of k centers to how good it is.

Different permutations are essentially the same solution.
However, Gradient descent does not know the objective function has symmetry unless we find some way to give this information to the algorithm.

A convex combination of 2 equivalent solutions is not optimal. The objective function has several equivalent global optima. 
%The fact that they're multiple optima is not a problem 
%rewrite
There is no way to tell gradient descent to be invariant to the permutation group.
%equivalence class? Optimize over manifold

If we are at a saddle point---ex. take the average of solutions $(x_1,x_2,x_3,\ldots$, $(x_2,x_1,x_3,\ldots)$, $(\fc{x_1+x_2}2,\fc{x_1+x_2}2,x_3,\ldots)$. The first two coordinates will stay the same.
%hard to define for manifold.
%first and second order that are not optimal, and method that works.

(Sometimes people reformulate with one global optima/do a convex relaxation but don't use this in practice because they would just like to use gradient descent.) 
%probability distribution over solutions.
\end{itemize}•
Plan:
\begin{enumerate}
\item
Geometry: identify property of objective function
\item
Algorithm: fast algorithm using the geometry
\item
Landscape: show specific problems have the geometric structure
%avoid case-by-case analysis
\end{enumerate}•
\subsection{Geometry: Strict saddle functions}

Problem: optimize a smooth function $f(x)$ with no constraints.
%second-order derivative Lipschits

We are worried about flat regions/saddle points (ex. the origin in $y=x_1^2-x_2^2$) and local optima. Local search can't get over local optima. So we ask:

For what objective functions are all local optima global optima?

%permutations

It can be easy to find a local min even with saddle points. \cite{ge2015escaping} Sometimes all local min are permutations of global min.

A strict saddle function is a function where all points are in one of three cases:
\begin{enumerate}
\item
near a local minimum (in a strongly-convex ball)
\item
near a saddle point (Hessian has negative eigenvalue)
\item
has a large gradient
\end{enumerate}

Example: $f(x) = \sumo in x_i^4 - \sumo in x_i^2$.
There are 4 equivalent local/global optima, $\pm(1,0),(0,1)$. Saddle points are $(0,0), \pm(1,1), \pm (1,-1)$.

This function can be used to solve PCA.

Other problems that satisfy strict saddle condition: 
\begin{itemize}
\item
eigenvector, generalized eigenvector
\item
some tensor problems   \cite{ge2015escaping} 
\item
dictionary learning [Sun Qu Wright 15]
\item
Community/synchronization [Bandeira, Boumal, Voroninski]
\item
Matrix completion  \cite{ge2016matrix} [Bhojanapalli, Neyshabur, Srebro 16]
\end{itemize}

Open problem: How can we modify objectives to make them satisfy the strict saddle condition?
Use re-parametrization, regularization, smoothing/homotopy. In homotopy, smooth by convolving with a Gaussian with large enough variance to make it convex. Then gradually reduce the radius of the Gaussian. Are there functions that have a bad optimization landscape but for which this helps? This works for a particular version of PCA.

What other geometric properties can we use?
\begin{enumerate}
\item
 Degenerate saddle \cite{anandkumar2016efficient} ``monkey saddle''---3rd order derivative is nonzero (it has 3 descreasing directions so a monkey can put its tail on the saddle), 
 \item
 most local minima are good \cite{choromanska2015loss}, based off spin-glass models.
\end{enumerate}•
For higher-order degeneracies, it can be NP-hard to even decide whether a point is a local min or a saddle.

With probability 1 a random function doesn't have a degenerate saddle point. Perturbing, we get rid of them, but there may be other local min.

What can we do on saddle points? 
\begin{itemize}
\item
Rely on second-order information. 
\item
Find the negative eigenvector, go along that direction.
\end{itemize}•
Problem: requires us to compute the Hessian. $d$ can be large; we have to compute a $d\times d$ matrix. Can we do this without computing the Hessian?

\begin{itemize}
\item
Frieze, Jerrum, Kannan 94, Learning linear transformations (specific objective function).  Considering first and second order conditions you can always find a global optimum
\item
Nesterov Polyak 03, Cubic regularization (needs 2nd order information)
\item
\cite{ge2015escaping} Stochastic gradient converges for strict-saddle function.
\item
Sun Qu Wright 15. Ridable functions, use trust region algorithm
\item
Lee Simchowitz Jordan Recht 16. Gradient descent also converges asymptotically. It doesn't bound the number of steps.
\end{itemize}
All of these assume the Hessian is Lipschitz. Assumptions are comparable. 


\subsection{Algorithm: Noisy stochastic gradient}

%only need one directoin.

Idea: saddle points are not stable. Gradient at $X$ gives no information, but gradients nearby approximates Hessian. Random walk around $X$ should discover the descending direction.
%stoch descent enuf?

Consider a quadratic function $f(x) = \rc 2 x^THx$, where $H$ is the (constant) Hessian. $(0,0)$ is a saddle point. Assume the axis is aligned, so $H$ is diagonal $\mattn{\la_1}000{\ddots}000{\la_d}$.
Assume $\la_d\le -\ga$ and for all $i$, $|\la_i|\le 1$. 

Start from $x^0\sim N(0,\ep I)$.  Usually $\eta=1/$smoothness; smoothness is a constant here so think of $\eta=\rc2$.
%1/smoothness.
\begin{align}
\nb f(x) &= Hx\\
x^{(t+1)}  &= x^{(t)} - \eta \nb f(x^{(t)}) \\
&= x^{(t) } - \eta Hx^{(t)}\\
&= (I-\eta H)x^{(t)}\\
x^{(t)} &= (I-\eta H)^t x^{(0)}.
\end{align}
In $\log (d)/(\eta\ga)$ steps the large direction blows up.
%initial proof relies on coupling


If objective method is quadratic function, this is a power method.

Recent work:
\begin{itemize}
\item
Agarwal et al 2016, Carmon Duchi Hinder Sidford 2016: find approximate local min in $O(\ep^{-1.75}\ln d)$ iterations. (Cubic regularization is $O(\ep^{-1.5})$, gradient descent (convex) is $O(\ep^{-2})$.) Try to solve the local problem by computing a Hessian-vector product.
\item
Jin et al 2017, Gradient descent with noise can also achieve $O(\ep^{-2} \poly(\ln d))$ for nonconvex functions.
\end{itemize}•

Tight analysis for gradient descent (Jin et al 2017): After adding noise, distance to saddle point is $\approx 1$ but correlation with negative direction is $d^{-.5}$. 
But the Hessian can be heavily distorted. (Ex. If only $\la_d$ is negative, for what points can I escape? As long as there is enough correlation with $e_d$, i.e. the point is not in a narrow band.  For a general function it's much harder to characterize which points can escape; the band of bad points is distorted.)
%some points no longer escape

If a point is stuck at saddle, then moving in a negative-eigenvector directoin, we can escape. A random perturbation whp takes us outside of the band.

We want faster algorithm in stochastic setting, simple accelerated algorithms (preferable stochastic). How to handle constraints?

%planted clique is a hard problem.

\subsection{Landscape: Matrix completion}

Matrix completion: Given an unknown low rank matrix $M$, observe a random subset of the entries of $M$. The goal is to recover $M$. Typical applications include recommender systems and collaborative filtering. 
Ex. factor a user-movie rating matrix (with missing entries) into user x Genre, Genre x movie matrices.

Previous works:
\begin{itemize}
\item
Convex relaxation (nuclear norm minimization): optimal sample complexity, but requires solving SDP.
\item
Non-convex optimization (most analyses is with good initialization)

Known: initialization is sufficiently good so that the function is convex locally.
\end{itemize}•
We show that even if you don't do initialization, nonconvexity is still not a problem!

Assume $M$ is of the form $M=ZZ^T$. Each entry is observed with probability $p$ independently. Let $\Om$ be the set of observed entries. The objective is
$$
\min_X \ub{\sum_{(i,j)\in \Om} (M_{ij} - (XX^T)_{ij})^2}{f(X)}.
$$
This is a degree 4 polynomial.

\begin{thm}
Understandard assumptions, with $\succsim dr^6$ observed entries, whp all local min are global minima.
\end{thm}
(Optimally, we we can hope for $dr$.)
Note that $f$ is not convex (there are saddle points) and $f$ has no degenerate saddle points.

%deterministic conditinos. expander plus?
%bhoj. incoherence <> expander

Warm-up: observe all entries
$$
g(X) =  {\sum_{(i,j)\in [d]\times [d]} (M_{ij} - (XX^T)_{ij})^2}.$$

How to go from this to random observation case? 
The eigenvectors could be very different in the partially observed matrix.
Note that linear statements are preserved by subsampling the entries of the matrix 
$$p\sum_{(i,j)\in [d]\times [d]} W_{ij}\approx \sum_{(i,j)\in \Om} W_{ij}$$
and combinations of linear statements are still preserved. (``Linear'' means that we have sums $\sum_{(i,j)}$; it does not have to be linear in the $W_{ij}$.)
This allows us to translate the full observation case proof to the partial observation case proof. Use concentration inequalities.

%thompson rounding? not rounding

Idea: Analyze the full observation case, redo step 1 by combining linear statements. Then proof generalizes to the partial observation case.
%if you launch a satellite and all parts are linear it is robust.

[Park Kyrillidis Caramanis Sanghavi 16] Asymmetrix matrix sensing. [Ge Jin Zheng 17] gives a framework for many matrix problems including asymmetric matrix sensing, completion, robust PCA with much simpler proof.

Open problems:
\begin{itemize}
\item
How to prove results on optimization landscape?
\item
Does over-parametrization help? (Hardt Ma Recht 16)
\item
Is the landscape robust against model misspecification?
\item
What is the optimization landscape for a random over-complete tensor?
\end{itemize}•

\subsubsection{Proof}

Rank 1 case for matrix completion: The objective function for the full observation case is $g(x) = \ve{M-xx^T}_F^2$. 

Lemma 1: $\nb g(x)=0$ implies $\an{x,z}^2=\ve{x}^4$. either $x$ is close to 0 or correlated with $z$.
$\nb g(x)=0$ gives the linear property
$$
\an{x,\nb g(x)}=0\implies \an{x,(zz^T-xx^T)x}=0.
$$

Translate the the partial observation case: $\an{x,z}^2 \approx \ve{x}^4$. Use concentration to get $\an{x, (P_\Om(zz^T)-P_\Om(xx^T))x}=0$. Do  concentration inequality over all $x$. (This is why you need at least $dr$ entries.)
%dr dimensions in lowrank matrices

Lemma 2: $\nb^2 g(x)\succeq 0$ implies $\ve{x}^2\ge \rc 3$. %\an{z,\nb^2g(x)z}$.

For the partial observation case, get $\ge \rc4$.

Combining these two lemmas, cor. 3: $x$ is local min implies $\an{x,z}^2 \ge \rc 9$ ($\ge \rc{16}$). More similar steps will show that local min is actually one of $\pm z$.


Matrix sensing: 
\begin{itemize}
\item
Unknown matrix $M^* = (U^*)(U^*)^T$ where $M^*$ is $d\times d$ and $U^*$ is $d\times r$, $r<d$.
%have one nonzero entry
\item
Known: sensing matrix $A_1,A_2,\ldots, A_m\in \R^{d\times d}$, $\an{A_i, M^*}=b_i$

Matrix completion is the special case that the sensing matrices are $E_{jk}$.
\item
Objective: 
\begin{equation}\label{eq:msense-obj}
\min_U \rc{2m} \sumo im (\an{A_i, UU^T}-b_i)^2.
\end{equation}
\end{itemize}
The RIP property:
\begin{df}
$\{A_1,\ldots, A_m\}$ is $(r,\de)$-RIP if for all $M$ or rank $r$, 
$$
\rc{m}\sumo im \an{A_i,M}^2 = (1\pm \de) \ve{M}_F^2.
$$
\end{df}
\begin{thm}[Srebro et al]
If $\{A_i\}$'s are $(4r, \rc{20})$-RIP, then all local optima of \eqref{eq:msense-obj}
%srebro and students
satisfy $UU^T = M^*$.
\end{thm}

Look at first and second-order optimality coneditions. Instantiate by taking the inner product with a gradient. Instantiate Hessian with some direction.

Intuition: We want to choose $\De=U-U^*$. 

Problem: $U,U^*$ are only unique up to rotations. 

Define  $\De = U-U^*R$. Let 
$$
R=\amin_{R\text{ is rotation}} \ve{U-U^*R}.
$$
A useful property (which relies on this rotation)
$$
\ve{\De\De^T}_F^2\le 2\ve{UU^T - M^*}_F^2.
$$
Rewrite the objective
$$
f(M) = \min_{\rank(M)=r} \rc{2m}\sumo im (\an{A_i,M}-b_i)^2.
$$
The benefit of this is that this is a quadratic function over $M$. It has a fixed Hessian. The Hessian is a $d^2\times d^2$ matrix but we can rewrite it as $f(M) = \rc 2 (M-M^*):H:(M-M^*)$ where $:$ here means to treat the left and right matrices as vectors and do vector-matrix-vector multiplication like a quadratic form.
So 
$$
M:H:N = \rc m \sumo im \an{A_i,M}\an{A_i,N}.
$$

Lemma: 
\begin{align}
\an{\nb f(U), \De}& = (UU^T - M^*) :H:(U\De^T + \De U^T)\\
\De : \nb^2 f(U) : \De &= (U\De^T + \De U^T) : H : (U\De^T + \De U^T) + 2(UU^T - M^*) : H : \De \De^T\ge 0.
\end{align}

Goal: replace everything here with either $A=UU^T-M^*$ or $B=\De\De^T$. Replacing $A+B = U\De^T + \De U^T$, 
\begin{align}
A:H:(A+B) &= 0\\
(A+B):H:(A+B) + 2A:H:B&\ge 0\\
B:H:B-3A:H:A&\ge 0\\
\De\De^T:H:\De\De^T - 3(UU^T - M^*) : H : (UU^T - M^*) &\ge 0.
\end{align}•
What does this mean? If $H=I$, this is just 
$$
\ve{\De\De^T}_F - 3\ve{UU^T-M^*}_F^2
$$
If $H$ is RIP, it is close to this. %\De\De^T 
%For matrix completion, 