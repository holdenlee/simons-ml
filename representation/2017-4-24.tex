\section{Multi-view Representation Learning with Canonical Correlation Analysis (Weirang Wang)}


Canonical correlation analysis (CCA) has been the main workhorse for
multi-view feature learning, where we have access to multiple
``views'' of data at training time while only one primary view is
available at test time. The idea of CCA is to project the views to a
common space such that the projections of different views are
maximally correlated.

In the first part of the talk, we compare different nonlinear
extensions of CCA, and find that the deep neural network extension of
CCA, termed deep CCA (DCCA), has consistently good performance while
being computationally efficient for large datasets. We further compare
DCCA with deep autoencoder-based approaches, as well as new variants.
We find an advantage for correlation-based representation learning,
while the best results on most tasks are obtained with a new variant,
deep canonically correlated autoencoders (DCCAE).

In the second part of the talk, we study the stochastic optimization
of canonical correlation analysis, whose objective is nonconvex and
does not decouple over training samples. Although several stochastic
optimization algorithms have been recently proposed to solve this
problem, no global convergence guarantee was provided by any of them.
Based on the alternating least squares formulation of CCA, we propose
a globally convergent stochastic algorithm, which solves the resulting
least squares problems approximately to sufficient accuracy with
state-of-the-art stochastic gradient methods. We provide the overall
time complexity of our algorithm which improves upon that of previous
work.

This talk includes joint work with Raman Arora (JHU), Jeff Bilmes
(UW), Jialei Wang (U Chicago), Nathan Srebro (TTIC), and Karen Livescu
(TTIC).

Goal: develop efficient learning algorithms for discovering structure of data and apply them to real-world applications.

\subsection{Multiview learning}

Deep learning methods require large amounts of labeled data. This is prohibitively expensive.

Training data consists of samples of $D$-dimensional random vectors that can be split into 2 subvectors $[x\in \R^{D_x};y\in \R^{D_y}]$.  For example,
\begin{itemize}
\item
audio + video
\item
audio + articulation
\item
image+ text
\item
text in different languages
\item
word + context words
\item
different parts of parse tree.
\end{itemize}
Extract useful features/subspaces in presence of multiple views at training time.

Ex. use multi-view learning on acoustics and articulation to get $\wt f:\R^{D_x}\to \R^L$. In ASR training, use the representation $\wt f(X_{train})$ to train a recognizer.

Ex. For example, shape of tongue tells you information not in the audio. These are not hard levels but can provide useful information. (Dataset: Wisconsin X-ray microbeam database. Metal pellets in mouth record movement.)
%Largest multiview 

Motivations: soft supervision, noise suppression, extract semantic subspace.

CCA: Given $N$ paired vectors $\{(x_i,y_i)\}_{i=1}^N$, find $(u,v)$
$$
(u,v) = \amax_{u,v} corr(u^Tx,u^Ty) = \amax_{u,v} \fc{u^T \Si_{xy}v}{\sqrt{(u^T \Si_{xx}u)(v^T \Si_{yy}v)}}.
$$
Extract further directions by repeating, subject to being uncorrelated with previous directions.

For extracting $L$-dimensional projections,
$$
\max_{U\in \R^{D_x\times L},V^{D_y\times L},U^T\Si_{xx}U = V^T\Si_{yy}V} \Tr(U^T\Si_{xy}V).
$$
Closed form solution is $T=\Si_{xx}^{-\rc 2} \Si_{xy} \Si_{yy}^{-\rc2}$.

Alternative formulation: Project 2 views to the same space.
$$
\min_{U,V} \rc N\ve{U^TX-V^TY}_F^2
$$

Interpretations: maximize mutual information, information bottleneck, latent variable model, discriminative properties.

Instead of applying linear function, use $f,g$ to extract features. Apply linear methods on top. 
\begin{align}
\wt f(x) &= U^Tf(x)\\
\wt g(y) &= V^T g(y).
\end{align}
%Not efficient.
Examples
\begin{enumerate}
\item
Kernel CCA: use RKHS feature mappings. Requires solving $N\times N$ eigensystem.
\item
Deep CCA: neural nets.
\item
Nonparametric CCA.
\end{enumerate}•

Optimization:
\begin{itemize}
\item
Gradient computation: wrt $f(X),g(Y)$ computed by SVD of $T=\Si_{ff}^{-\rc 2} \Si_{fg}\Si_{gg}^{-\rc 2}$. 
\item
Gradients wrt $W_f,W_g$ from standard backprop.
\item
Objective is not expectation of loss over examples  due to constraints $U^T\pa{\rc N \sumo iN f(x_i)f(x_i)^T}U=I$.
\item
Stochastic optimization with large minibatches works well.
\end{itemize}

DCCA performs best.

CCA-based features give large improvements in phonetic recognition. 
DCCA scales better than approximate KCCA.

Possible issues:
\begin{itemize}
\item
Lose private but useful information in primary view.
\item
Difficult for SGD due to constraints.
\item
Consider alternatives based on autoencoder regularization.
\end{itemize}

Models:
\begin{enumerate}
\item
DCCAE: add reconstruction objective
\item
CorrAE: 
each dimension has scale 1 but we don't enforce different dimensions to be uncorrelated. $u_i^T f(X)f(X)^Tu_i = N$, same for $v$. This gives a large degradation in performance.
\item
SplitAE: 
Assume common underlying signal can be extracted from one view and used to reconstruct both views.
\end{enumerate}•
Noisy MNIST dataset with rotations. The second view has random background noise. %For each image, the second view has random background.

%The rotation degree of freedom is not shared.
Classification error rate of linear SVM on projections. DCCAE gets $2.2\%$.

Canonical correlation-based objectives typically outperform autoencoder-based and correlation-based objectives especially when inputs are noisy. Whitening constraint is important (projected data has 0 mean, different directions uncorrelated).

Latent variable interpretation of CCA: $z\sim N(0,I)$, $x|z,y|z\sim N(Wz,\Si)$. Max likelihood in CCA space.

Variational CCA: parametrize $p(x|z),p(y|z)$ with DNNs. Parametrize approximate posterior $q(z|x)$ with another DNN, and maximize variational lower bound of log-likelihood. 
$$-KL(q_\phi(z|x)||p(z)) + \E_{q_\phi(z|x)} [\ln p_\te p(x|z) + \ln p_\te(y|z)].$$
Sampling: use MCMC.

There can be large variation not captured by shared variables. Solution: 
Incorporate private variables for better reconstruction and disentanglement of variations.

VCCA-private does best in mAP (mean average percision) on MIR-Flickr.

Optimization: Add $\ga_xI$ to $\Si_{xx}$ and similarly for $y$ for regularization. Closed form solution $T$ but memory/time consuming.
%, optimum achieved by $(\Si_{xx}^
%matrix/vector optimization.
Do alternate least squares. Cf. two-sided power method.
%$\matt O{T}{T^T}O [\phi_{t-1};\psi_{t-1}]$

Assuming singular value gap $\rh_1>\rh_2$, measure alignments $(\phi_t^T \phi)^2$ and $(\psi_t^T\psi)^2$. In $T=O\pa{\fc{\rh_1^2}{\rh_1^2-\rh_2^2}\ln \prc{\eta}}$, produces $\eta$-suboptimal solutions.

The matrix-vector multiplication is equivalent to solving the least squares problem.

Inexact power iterations with approximate matrix/vector multiplications still converge. 

Derive meta-algorithm: solve least squares to $\ep$-suboptimality instead. 

If error is $\ep(T)=O(\eta^2)$, converge in about the same number of iterations.

LS has finite sum structure so this is a well-studied problem, many stochastic algorithms converge linearly. Use stochastic gradient corrected with the batch gradient.

More efficient: shift-and-invert.  Run power iterations on $(\la I - C)^{-1}$. Gap is larger. Requires smaller number of outer iterations.

%discussion: 100 dim random fourier features.
%Noisy power iteration is differentially private. Streaming shift-and invert.