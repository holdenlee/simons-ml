\def\filepath{C:/Users/oldhe/Dropbox/Math/templates}

\input{\filepath/packages_article.tex}
\input{\filepath/theorems_with_boxes.tex}
\input{\filepath/macros.tex}
\input{\filepath/formatting.tex}

\pagestyle{fancy}
\lhead{Simulated annealing and IPM}
\chead{} 
\rhead{} 
\lfoot{} 
\cfoot{\thepage} 
\rfoot{} 
\renewcommand{\headrulewidth}{.3pt} 
\setlength\voffset{0in}
\setlength\textheight{648pt}

\addbibresource{bib.bib}

\begin{document}

Our goal is to solve
$$
\min_{x\in K} \wh \te^T x
$$
where $K$ is convex and compact.

\section{Interior point methods}

Path-following interior point is the following:
\begin{itemize}
\item
While $t>\ep/m$,
\begin{itemize}
\item
``Approximately'' solve 
$$\wh x_t \approx \amin_{x\in K} \ub{x^T\wh \te + t\ph(x)}{\Phi_t(x)}$$
 where $\ph$ is self-concordant barrier function.
\item
$t\mapsfrom t(1-\ga)$.
\end{itemize}•
\end{itemize}
%Requirements
Examples for $\ph$:
\begin{itemize}
\item
$K=\set{x}{Ax<b}$. Then $\ph_K(x) = \sum_i -\ln (b_i - A_ix)$ works.
\end{itemize}
Requirements:
\begin{enumerate}
\item
$\ph\to \iy$ as $x\to \pl K$.
\item
$\nb_x^3 \ph[h,h,h] \le 2(\nb_x^2 \ph[h,h])^{\fc 32}$
\item
$\nb_x \ph[h]\le \sqrt{\nu\nb_x^2\ph[h,h]}$.
\end{enumerate}

Approximate algorithm: Newton's method
$$
\wh x^+\mapsfrom \wh x - c \nb^{-2} \ph(\wh x) \nb \Phi_t(\wh x).
$$
Analysis relies on 
$$
\la_t(x) = \sqrt{\nb \Phi_t(x) \nb^{-2}\ph(x) \nb \Phi_t(x)}
$$
being small. I want $\nb \Phi_t$ to be small measured in norm of Hessian. 

%how far off did I make it worse.
\begin{align}
t^+ &:= t\pa{1-\fc{d}{\sqrt \nu}}\\
\la_{t^+}(\wh x) &\le \la_t(\wh x)(1+c)+c\\
\la_{t^+}(\wh x^+) &\le 2\la_{t^+}^2(\wh x).
\end{align}
%
%alm grad of new obj
%once newton decrement $<\rc2$, then decays quadratically.
$\la$ doesn't increase too much when you increase the temperature a little; then after doing a Newton step it decreases quickly.

Once $t<\ep$ you have an $\ep$-approximate solution. 

%draw balls around the Hessian.

We need $(1-\fc d{\sqrt \nu})^L<\ep$, so need $K=\Te(\sqrt \nu \ln \prc\ep)$ epochs.

%Nesterov Nemivroski Russia to US

Does there exist $\ph$ for every compact convex $K$?
Nesterov and Nemivroski showed yes, for $\nu=O(n)$. It is constructive but not efficiently computable (volume of polar cones...)

\section{Simulated annealing}

\begin{itemize}
\item
While $t>\ep$,
\begin{itemize}
\item
Collect $N$ (approximate) samples $X_t^1,\ldots, X_t^N\sim P_{\te/t}$ where $P_{\te_t(x)} \propto \exp(-\fc{\te^Tx}{t})$ on $K$.
%t put more weight on those with smaller value.
\item 
Update $t\mapsfrom t(1-\rc{\sqrt n})$.
\item
Use samples $X_t^1,\ldots, X_t^n$ to ``warm start'' on new point.
%random walk around some procedure
\end{itemize}•
\end{itemize}•
For Hit-and-Run you need warm start and the covariance matrix. Take $N\approx n$ to get good estimate on covariance matrix (isotropically chose). 

Hit-and-run:
\begin{itemize}
\item
Input: $\Si, X_0$ warm start, $P_0$ (oracle access)
\item
For $k=1,2,\ldots,K$,
\begin{itemize}
\item
Sample $U\sim N(0,\Si)$
\item
Compute ray $R=\set{X_{k-1}+\al U}{\al\in \R}$.
\item
Sample $X_k \sim P_\te|_R$.
\end{itemize}•
\item
Return $X_K$.
\end{itemize}•
Here I only need to be able to check whether points are in the set!

Lovasz, Vempala, etc.: HAR$(\Si, X_0,P_0)$ returns $X$ such that $X$ is approximately distributed according to $P_\te$, given
\begin{enumerate}
\item
$X_0$ is sampled from ``nearby'' distribution
\item
$\Si$ is ``almost'' $\Cov(P_\te)$.
\item
$K=cn^3$.
%close in KL diverges
\end{enumerate}
%In every step of HAR
%3+1+0.5
%n samples per epoch, $\sqrt n$ epochs

\begin{thm}
If $X\sim P_{\te/t}$, then $\E[\wh \te^T x]\le \min_\al \wh \te x + tn$.
\end{thm}

Abernethy-Hazan 2016: There exists a strongly convex barrier function $\ph$ such that 
\begin{align}
\amin_{x\in K} \wh\te^T\ph(x) + t\ph(x) &=
\EE_{x\sim P_{\te/t}}[x]
\end{align}
%newton to move. update tem. upeate 
%1/\qs]sqr
%sampe


What is $\ph$?
For exponential families
\begin{align}
A(\te) & = \ln \int_K \exp(-\te^T x)\dx\\
\ph(\te) = A^*(\te)&= \sup_\te\te^Tx - A(\te),
\end{align}•
Fenchel conjugate of $A$,
\begin{align}
P_\te(x) &= \exp(-\te^T x - A(\te))\\
\nb A(\te)& = \EE_{X\sim P_\te}[X].
\end{align}•
(Check signs.)
%derivative of function 0 is maximizedatlll
%constructed so work out

Fact:
$$
\nb A(\te) = \amax_{x\in K} x^T\te - A^*(x).
$$
The last 2 equations explain why this works.

Q: What's the barrier parameter that this gives us? It is $n$ all the time, but this is misleading: you can make parameter better by scaling.

HAR gives the following entropic barrier for PSD cone: 
$$
\ph(X) = n\ln (\det X).
$$
(We can divide this by $n$.)
%pos orthant, PSD, lorentz cone same.
%symmetric
%entropic barrier. 

%If you can show the constant becomes better after scaling, you can do better. 

Kalai-Vempala give time $n^{4.5}$, we show time $n^4\sqrt\nu$.

%RHS odeons't defffff
%tricks aready shown using ...

%subset of psd cone. Can increast tem

Cf. Narayanan---Dikin walk---single step every round, compute covariance at each step.
%compute hessians
%much better 
%implicitly a barrier function.

\printbibliography
\end{document}