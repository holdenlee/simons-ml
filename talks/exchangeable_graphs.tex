\def\filepath{C:/Users/oldhe/Dropbox/Math/templates}

\input{\filepath/packages_article.tex}
\input{\filepath/theorems_with_boxes.tex}
\input{\filepath/macros.tex}
\input{\filepath/formatting.tex}

\pagestyle{fancy}
\lhead{Exchangeable graphs}
\chead{} 
\rhead{} 
\lfoot{} 
\cfoot{\thepage} 
\rfoot{} 
\renewcommand{\headrulewidth}{.3pt} 
\setlength\voffset{0in}
\setlength\textheight{648pt}

\addbibresource{bib.bib}

\begin{document}

The Class of Random Graphs Arising from Exchangeable Random Measures 
Victor Veitch and Daniel M. Roy 

\url{https://arxiv.org/abs/1512.03099}

Sampling and Estimation for (Sparse) Exchangeable Graphs 
Victor Veitch and Daniel M. Roy

\url{https://arxiv.org/abs/1611.00843}

And maybe also this survey:

Bayesian Models of Graphs, Arrays and Other Exchangeable Random Structures 
(with Peter Orbanz) 
IEEE Trans. Pattern Anal. Mach. Intelligence (PAMI), 2014. 

\url{https://arxiv.org/abs/1312.7857}

The talk will be informal but based on slides I've produced for previous talks. Here's a recent abstract:

Statistical models of sparse networks from symmetries

The statistical analysis of network data rests on a foundation of random graph models, yet existing models are inadequate because they cannot model sparse graphs or fail to meet basic statistical criteria. We introduce a new class of random graphs, defined by the exchangeability of their vertices, and indexed by the expected number of edges, rather than the number of vertices. A straightforward adaptation of a result by Kallenberg yields a representation theorem: every such random graph is characterized by three (potentially random) components: a nonnegative real I in $\R_+$, a measurable function $S: \R_+ \to \R_+$, and a symmetric measurable function $W: \R_+^2 $ to $[0,1]$, where both $S$ and $W$ satisfy weak integrability conditions. We call the triple $(I,S,W)$ a graphex, in analogy to graphons, which characterize the (dense) exchangeable graphs indexed by the cardinality of their vertex sets. I will present some results about the structure and consistent estimation of these random graphs, and what role they can play in the analysis of real-world networks.

I'm coming from a statistical modeling perspective. What is an observation? What is the model?

We need a framework for the statistical analysis of networks:
\begin{itemize}
\item
family of distribution on large graphs (populations)
\item
distribution over observed (sub)graphs (samples)
\end{itemize}•
These are often conflated.

Real world networks are sparsely connected. 
\begin{itemize}
\item
Random graph models should be sparse, $o(n^2)$ deges among $n$ vertices.
\end{itemize}•
Problem: 
There is no suitable general framework. 
\begin{itemize}
\item
Graphons cannot get you this. The problem is exchangeability.
\item
Do we need to abandon exchangeability to get sparsity? No. (Caron and Fox 2014, Kallenberg 1990)
\end{itemize}•
We introduce a general class of random graphs suitable for modelling sparsely connected network structures. The math structures we exploit are exchangeable random measures. 

Special cases:
\begin{itemize}
\item
dense exchangeable graphs (graphon model)
\item
Caron and Fox's nonparametric Bayesian models of sparse graphs
\item
Sparse graphs with mall world, power law behaviors.
\end{itemize}•
Preferential attachment yields these.

\begin{itemize}
\item
Exchangeability can be related to a sampling process.
\item
Sampling process is perverse for large (sparse) graphs.
\end{itemize}

Example: 
\begin{itemize}
\item
Initialize: sum 1, total 2.
\item
Next draw: Bernoulli sum/total. Add y to sum, 1 to total.
\end{itemize}
The probability of  a path depends only on where it ends; all that matters is the total counts of 0's and 1's.

There's another description of this process, identical at the level of what numbers pop out.
\begin{itemize}
\item
Let theta be uniform from $[0,1]$.
\item
Return bernoulli(theta).
\end{itemize}
Note this doesn't include mutation.
%slope arctan

A sequence of random variables $Y$ is exchangeable when
$$
(Y_1,\ldots, Y_n) \stackrel d= (Y_{\pi(1)},\ldots, Y_{\pi(n)})
$$
for all $n\in \N$, permutation $\pi$ of $[n]$.

\begin{thm}[De Finetti]
TFAE:
\begin{itemize}
\item
$Y$ is exchangeable.
\item
$Y$ is conditionally iid given some $\te$
\item
Exists $f$ such that
$$
Y \stackrel d (f(\te,U_1),f(\te,U_2),\ldots)
$$
for iid uniform $\te, U_1,U_2,\ldots$.
\end{itemize}•
\end{thm}
Let's say $n$ is large; look at a prefix $k\ll n$. That sequence is exchangeable. Insofar as $n$ grows much larger than $k$, it looks like it is the draw from a structure like this.

Given a sample, I need to think it's a prefix of an infinite exchangeable sequence.

Represent a graph as an adjacency matrix, or as a function on $[0,1]^2$, ``graphical graphon''.

Joint exchangeability on an array $X$ on $\N^2$ means
$$(X_{ij})\stackrel d=(X_{\si(i)\si(j)}).$$


Fix $n\in \N$, graphon $W:[0,1]^2\to [0,1]$ symmetric, measurable. Generative model for $G(n,W)$:
\begin{enumerate}
\item
Let $\xi_1,\xi_2,\ldots$ be iid uniform $[0,1]$.
\item
Connect vertices $i$ and $j$ for $i<j\le n$, independently with probability $W(\xi_i,\xi_j)$.  
\end{enumerate}•
Write $G(W)=G(\iy,W)$.

\begin{thm}[Aldous-Hoover]
Let $X$ be infinite symmetric adjacency matrix. TFAE:
\begin{enumerate}
\item
$X$ is jointly exchangeable.
\item
$X$ is $W$-random graph for some (potentially random) graphon $W$.
\end{enumerate}
\end{thm}

Making the graph larger and larger, viewed from the sampling processes perspective, it's possible that the law of the sampling process will converge. The sequence of graph converges to a measurable function.


What can you learn about facebook by sampling neighborhoods? That's an open problem.

We address sampling with a uniform sampling. It gives relative frequency of subgraphs. Statistically it's a little perverse because we don't sample this way.

Ex. movie ratings: force people to watch movies. Expect underlying ratings to satisfy exchangeability.

%To get a statistical model, 

Let $W,W_1,W_2,\ldots$ be a sequence of graphons. Write $W_n\to_{\square} W$ if $G(W_n)\to G(W)$ in distribution. Through random sampling they produce similar subgraphs.

%squint
%cut metric
\begin{thm}[Kallenberg 99]
Let $\wh W_n$ be empirical graphon of $G(n,W)$. Then $W_n\to_\square W$ a.s.
\end{thm}
Multivariate sampling and the estimation problem for exchangeable arrays.

%other metrics not equivalent

\begin{cor}
Exchangeable graphs are dense or empty a.s.
\end{cor}
Use linearity of expectation and exchangeability, 
$$
\E\pat{edges} = \binom n2 \E[X_{1,2}].
$$

Caron and Fox 2014: go beyond dense structures. 
Each vertex gets a real-valued label. An edge between $1$ and $2$ means a point at $(\te_1,\te_2)$. We can think of the points as a set in $\R_+^2$. We've lost the disconnected vertices. A random sample from a sparse graph has many disconnected vertices.

Call this an adjacency measure: each point $(\te_i,\te_j)\in \R_+^2$ is an edge $ij$.

Kallenberg exchangeable graph (Veitch-Roy 15)

Fix a size $s\in \R_+$. Fix a graphex $W:\R_+^2\to [0,1]$ symmetric, measurable. 

Generative model for $K(s,W)$, 
\begin{itemize}
\item
Let $\Pi=\sum_i \de_{(\te_i,\xi_i)}$ be a unit-rate Poisson process on $[0,s]\times \R_+$. 
\item
For every edge $i<j$, $i,j\in \N$, add edge $(\te_i,\te_j)$ with probability $W(\xi_i,\xi_j)$. 
\end{itemize}
Also star structures and isolated edges...

Sample from an infinite strip $[0,s]\times \R$. Decide whether to put an edge $(\te_i,\xi_i), (\te_j,\xi_j)$ by flipping $W(\xi_i,\xi_j)$. I forget about the isolated vertices.

It is like the limit of a graphon as it gets more sparse.

How to adjust to do preferential attachment? If you run preferential attachment for a long time and choose a vertex at random it looks locally treelike. You won't see that here; you see isolated edges.

%prob edge depend on what seen so far.
%make $W$ random and integrate it out. Look at previous edges to see how to add new. Preferential attachment can't get exactly satisfy . Rich get richer.

If $W$ is integrable, in expectation you get finitely many edges. 

%huge subgraph conntected.

Let $X$ be infinite adjacency measure. TFAE:
\begin{enumerate}
\item
$X$ is jointly exchangeable.
\item
$X$ is Kallenberg exchangeable graph for some (potentially random) graphex $W$. 
\end{enumerate}
%don't dial in number of indiv, expected number of individuals.

At particular times, structure shows up.

Graphons can be represented as graphexes. Define on square of size $c\times c$. 
Let
$$
W(x,y) = \begin{cases}
\wt W(x/c,y/c),&x,y\le c\\
0,&\text{else}.
\end{cases}
$$
Call $W$ a $c$-dilation of $\wt W$. 

Generative model:
\begin{align}
N_s &\sim Poi(c\cdot s)\\
\{\te_i\}&\sim Uni[0,s]\\
\{\xi_i\}&\sim Uni[0,1]\\
G_s\{(\te_i,\te_j)\}|\xi_i,\xi_j &\sim Bernoulli (\wt W(\xi_i,\xi_j)).
\end{align}•

How many edges, vertices do we expect to see?
\begin{align}
\E[e_s] &= \rc 2 s^2 \ve{W}_1.
\E[v_s]& = s\int_{\R_+} (1-e^{-s\mu_W(\la)})\,d\la.
\end{align}
$\E [v_s]$ depends on tail.

Ex. polynomial tail power law. Exponential tail in middle. Compactly supported dense.

For example, $W(x,y) = (x\ne y) (x+1)^{-2}(y+1)^{-2}$ gives 
\begin{align}
\E[e_s] &= s^2\\
\E[v_s]&= \sfc{\pi}3 s^{\fc 32}\\
P(D_s=k|G_s)&\stackrel p{\to}\fc{\Ga(-\rc 2+k)}{2\sqrt \pi k!} k^{-\fc 32}
\end{align}

How to estimate a graphex from observed network? Have same notion of convergence.

Define empirical graphex to be $\fc{v_s}s$-dilation of $\wt W_s$. 

\begin{thm}
Let $\wh W_s$ be empirical graphex of $K(s,W)$. Then $\wt W_s\to_\square W$ in probability.
\end{thm}
%convergence of comb sequence

%What does exchangeability mean intuitively? 
If people's probabilities of connecting are only determined by latent variables, you get a dense graph.
Will I ever cross paths for you (history dependence)---this is sparse. Subject to possibility there's a dense exchangeable thing going on. 

We've punted on the question of how to model large graphs. The latent variables have to be baked into the $W$'s.

Ex. $W(x,y) = f(x)f(y)(x\ne y)$ models popularity. 
%cities

%Applied to a large grid you lose the grid structure.

Reason about change (dynamics) of graphs. There are many details in sequence model.

Clustering coefficient? (If two people have a common friend they are more likely to be friends.) You can create the property or make it impossible through $W$.



%homophily)
\printbibliography
\end{document}