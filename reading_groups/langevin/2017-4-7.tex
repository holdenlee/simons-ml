\section{A Semiclassical Approach to the Kramers–Smoluchowski Equation}

(Probability/PDE seminar)

We consider the Kramers–Smoluchowski equation at a low temperature regime and show how semiclassical techniques developed for the study of the Witten Laplacian and Fokker–Planck equation provide quantitative results. I will give complete proofs in the one dimensional case and explain how recent work of Laurent Michel gives results in higher dimensions. One purpose is to provide a simple introduction to semiclassical methods in this subject.

The equation is
\begin{align}
\pl_t \rh &= \pl_x (\pl_x \rh - \ep^{-2} \rh \pl_x\ph)\\
\rh|_{t=0}&= \rh_0.
\end{align}
$\ph$ is a Morse function on $\R$: it has an isolated nondegenerate critical point. It goes to $\iy$ at $\iy$.
(This is a Witten Laplacian.) %The existing work did not provide

Let $S=\ph(s)$ be the height at $s$. Assume the Hessians at the 2 local minima (zeros) are $\ph''(m_1)=\ph''(m_2) = 1 = -\ph''(s)$. 

Based on works: Evans-Tabrezian 16 and Seo-Tabreziam 17.

%Suppose we have perturbed 
Define $E_1,E_2$ by
$$
\set{x}{\ph<S} = E_1\sqcup E_2.
$$
\begin{thm}
Suppose 
$$
\rh_0 = \prc{2\pi \ep^2}^{\rc2} (\be_1\one_{E_1} + \be_2\one_{E_2} + v_\ep) e^{-\fc{\ph}{\ep^2}}
$$
with $\ve{v_\ep}_\iy\to 0$.
(``well perturbed'') (Think of Gaussian concentrating in each well.)
Then in the distribution sense
$$
\rh(2\ep^2 e^{S/\ep^2}\tau, x) \xra{\cal D'(\R)} \al_1(\tau) \de_{m_1}(x) + \al_2(\tau) \de_{m_1}(x)
$$
%scaled
uniformly for $t\ge 0$, where
$$
\pl_\tau \al = -A_0\al, \quad \al(0)=\be, \quad A=\smatt 1{-1}{-1}1.
$$
\end{thm}

Laurent Michel
%quote from 2nd ed Hardy's pure math.

What if I take arbitrary initial data? The same is true with error term in time; you have to wait a certain time before it get small.

Higher dimensional: 
Assume we have Morse function $\ph:\R^n\to \R$, 
\begin{align}
0&=\inf_{x\in \R^d} = \ph(m_j)&m_j&\in U^{(0)}\\
S&=\sup_{S\in U^{(1)}} = \ph(s_j)&s_j&\in U^{(1)},
\end{align}
$\la_1^2(s)/\ph''(s) = -1$, $\la_1(s) \in \Spec(\ph''(s))$, $\la_1<0$.

Here $U^{(i)}$ is the set of critical points of index $i$. 
Index 0 means no negative eigenvalues. Index is number of negative eigenvalues.

Suppose 
$$
\{\ph<S\} = \bigsqcup_{j=1}^N E(m_j),
$$
indexed by the minima the sets contain.
The fact that this is a Morse function means 
$$
m, m'\in U^{(0)}, \quad m\ne m' \implies \ol{E(m)}\cap \ol{E(m')} \subeq U^{(1)}.
$$
For all $s\in U^{(1)}$, there exists $m,m'$, $s\in \ol{E(m)}\cap \ol{E(m')}$.

Make the simplifying assumption $m',m\in U^{(0)}$, $m\ne m'$ implies $\ol{E(m)}\cap \ol{E(m')}$. %they intersect
(We assume same depth of all the wells and same height of the saddle points. The statement becomes more complicated if this is dropped.)

Form the graph
$$
\cal G = (V,E) = (U^{(0)}, U^{(1)})
$$
where if $s\in U^{(1)}$ is an edge between $m,m'$, then $s\in \ol{E(m)}\cap \ol{E(m')}$.  In dimension 1, we get a single edge. In 2 dimensions, e.g. we can have a triangle.

See Landim et al 15.

In the theorem statement, let
$$
\rh_0 = \sum (\be_n \one_{E_n} + v_\ep) e^{-\ph/\ep}
$$
%weighted graph laplacian
where $\al_\tau = -A\tau$, $A$ the graph Laplacian of $\cal G$, defined by 
$$
A(m,m') = \begin{cases}
d(m),&m=m'\\
-1,&m\ne m' \text{ connected by edge}\\
0,&\text{otherwise}.
\end{cases}•
$$
where $d(m)$ is the degree of vertex $m$.

For example the Laplacian of the triangle is $\mattn 2{-1}{-1}{-1}2{-1}{-1}{-1}2$.

References: Witten 82 (Laplacian approach), Helffer-Sjo'strand,... Bovier-Gajnard-Klein (probabilistic, spectral approach), Helffe-Nier.

I can have a chain if the height and minima are the same.

The operator is
\begin{align}
P &= \pl_x(\pl_x + \ep^{-2} \pl_x\ph\pl_x)\\
&= \pl_x e^{-\fc{2\ph}{h}}\pl_x e^{2\ph/h}\\
h&=2\ep^2\\
e^{\ph/h} P e^{-\ph/h}&=-h^{-2}\De \ph\\
\De \ph&=-h^2\De + |\pl_x \ph|^2 - h\De \ph
\end{align}
Witten Laplacian on $f$'s. 
You get a lot of mileage from supersymmetric structure of this equation.
%all this can be done on Riemannian manifold. this is on flat space.
\begin{align}
\pl_t\rh &= P \rh\iff u_t = -\De \ph u, &u(t,x) &= e^{\ph/h} \rh(h^2 t,x)\\
\psi_n(x):&= c_h(h) h^{-\rc 4} \one_{E_h}(x) e^{-\ph(x)/h}, & \ve{\psi_h}_{L^2}&=1\\
\Psi(\be) &= \sum_h \be_h \Psi_h.
\end{align}
\begin{thm}
There exists $A(h)=A_0+hA_1+$ such that (here $A_0=\smatt 1{-1}{-1}1$)
$$
\ve{e^{-t\De\ph} \Psi(\be) - \Psi(e^{-t he^{-s/h} A_0\be})}_2\le Ch |\be|,\quad t\ge 0.
$$
%change coeff according to this
\end{thm}

The Witten Laplacian. 
%I  this how you guys do Morse theory?

Differentiation takes $k$ forms into $k+1$ forms. In 1-dimension there are only 0 and 1 forms.
$$
d:C^\iy (M,\Om^k) \to C^\iy(M, \Om^{k+1}).
$$
Let $g$ be the metric on $M$. 
We can take the adjoint of this operator.
$$
d^* : C^\iy(M, \Om^{k+1}) \to C^{\iy} (M,\Om^k).
$$
The Hodge Laplacian is
\begin{align}
\De: C^\iy(M, \Om^k) \to C^\iy(M,\Om^k).
\end{align}
(Write $\De^{(k)}$ to emphasize that it acts on $k$ forms.)
On functions, $d^*=0$. 
%don't have the first on 1-D

Consider the deformation
$$
d_\ph = e^{-\ph/h} h d e^{\ph/h}.
$$
%change to volume form

A quick calculation shows
$$
\De_\ph = d_\ph^* d\ph.
$$
A basic fact: $\De_\ph^{(k)}$ has $|U^{(k)}|$ eigenvalues in $[0,\ep_0h]$. The number of small eigenvalues is given by the number of critical points of index $k$.

Here is an immediate application, and motivation. 

We have an easy Morse inequality. Let $M$ be a compact manifold. Then
$$
\dim H^{(k)}(M, \R)\le |U^{(k)}|.
$$
Hodge theory shows that this is equal to $\dim\set{u\in C^\iy(M,\Om^k)}{\De_\ph h=0}$, harmonic functions. 
%0 
%witten's proof of Morse: 1-line

\begin{proof}
For $k=0$, $-h^2\De  + |\pl \ph|^2 - h\De \ph$, 
\begin{align}
\an{\De_\ph h, u} &\ge \ep h\ve{u}_{L^2},
\end{align}•
$u\in V^{\perp}\cap Dom(\De\ph)$, $\dim V = |U^{(0)}|$. Do this by harmonic approximation %as in quantum mech,
$$
\mathfrak S(-h^2\De + x^2) =\set{(2|n|+1)h}{n\in \N_0}
n\in \N_0
$$
where $n=n_1+n_2+\cdots$.
Near a minimum, approximate by harmonic oscillator.
$$
-h^2 \De + |\pl \ph|^2 - h\De \ph \approx -h^2 \De + |\ph''(m)(x-m)|^2 - h\De\ph(m).
$$

We show there are $|U^{(0)}|$ approximate eigenvalues. 
%for self-adjoint operator
%cancel, eig 0.

%basic spectral theory
%if 2 orthogonal, have 2 eigenvalues

%\subsection{Quasimodes}
Quasimodes: These are approximate eigenvalues. 
$$
f_n^{(0)} = h^{-\rc 4} C_n(h) \chi_h(x) e^{-\ph/h}.
$$
(Note 0 eigenvalue: $e^{-\ph}$.)
We have $\an{f_n^{(0)}, f_m^{(0)}} = \de_{mn}$, $\ve{f_h^{(0)}}=1$.
\begin{align}
d_\ph &= \te(h\pl_x + \ph'(x))\\
\De_\ph f_h^{(0)}&= \te(e^{-(S-\ep)/h}).
\end{align}•
%modulo doing for all forms.
%there's no hard part
%wait until... compute precisely without using supersymmetric structure. People tried and faild.
Supersymmetric structure means writing in terms of $\De_\ph$. 
%insightful part.
``The hard part is to come up with the easy way of doing it.''
\end{proof}
This proves the basic fact.

%lorentz paper, 
For different heights: 60 pages, combo in higher dimensions. Complicated labeling procedure, which saddle points are associated with which minima. 

%Eigenvalues $he^{-2S/h}$ very different from estimate. This precise eigenvalue appeared.

%construction of good quasimodes
%5 pages to do. francesco wrote in 5.
%result more precise. for finite times, here for all times. 
%here get asympt expansino of everything. 
%finite many terms, exp decay.
$$
d_\ph^* = d_{-\ph}, \quad \De\ph^{(1)} = \De_{-\ph}^{(0)}.
$$
%dimension 1

We construct quasimodes.
\begin{align}
f_{(x)}^{(1)} &= h^{-\rc 4} d_h(h)\te(x) e^{-(S-\ph(x))/h}\\
d_h(h)&\sim \pi^{-\rc 2}+\cdots\\
\ve{f^{(1)}}_{L^2} &=1\\
d_\ph^* f^{(1)} &= \te(e^{-\al/h})\\
\De_{-\ph} f^{(1)} &=\te(e^{-\al/h}).
\end{align}•
%5 pages
Now construct 2 projectors $\Pi^{(0)}$, $\Pi^{(1)}$. 
\begin{align}
g_h^{(0)}  &= \Pi^{(0)} f_h^{(0)}\\
g^{(1)} &= \Pi^{(1)} f^{(1)} .
\end{align}
Use Gram-Schmidt to orthonormalize, get $\{e_n^{(0)}\}$ and $e^{(1)}$.  

These are close to the original. Change-of-basis is close to identity modulo exponentially small errors. 
\begin{align}
e_n^{(0)} - f_n^{(0)} &=O_{L^2} (e^{-(S-\ep)/h})\\
e^{(1)} - f^{(1)} &= O_{L^2} (e^{-\al/h}).
\end{align}•
\begin{align}
\De_{-\ph} \circ d_\ph &= d_\ph\De_\ph\\
E^{(j)}&=\Pi^{(j)} L^2\\
d_\ph^*\De_{-\ph} &= \De_\ph d_\ph^*\\
\implies
d_\ph(E^{(0)}) &< E^{(1)}\\
d_\ph^*(E^{(1)}) &< E^{(0)}.
\end{align}

We want the matrix of operator $d_\ph$ on this basis.
\begin{align}
\cal L &= d_\ph |_{E^{(0)}\to E^{(1)}}\\
\cal L^* &= d_\ph^*|_{E^{(1)}\to E^{(0)}}\\
M&=\cal L^*\cal L = \De \ph|_{E^{(0)}}.
\end{align}
(When wells are different depths, exponential factors have different sizes. Look at characteristic values of individual $L$'s. Shift exponentially small things.)
%can be approx by matrix expansion
%supersymmetric structure, get $he^{2Sh}$

We compute $L$, the matrix of $\cal L$ in $e_h^{(0)}$.
\begin{align}
\wh L_j &= \an{f^{(1)}, \pl_\ph f_j^{(0)}} = h^{-\rc 2} d(h) c_j(h) \int_\R \te(x) e^{-(S-\ph)/h} d_\ph(\chi_j e^{-\ph/h})\\
%1x2
&=h^{\rc 2} d(h)c_j(h)  e^{-S/h} \int \te(x)\chi_j'(x)\dx = (-1)^j h^{\rc 2} e^{-S/h} (\pi^{-\rc 2}+\cdots)\\
\wh L &=\pi^{-\rc 2}he^{-S/h}  ([-1,1]+ hA_1+\cdots  + \te(e^{-(S+\al)/h})).
\end{align}
%numbers exist in platonic reality. 
(If I had analytic things, I know when to stop to get exponentially small error.)

Choose $\te$ so there is right overlap. We had $e^{-(S-\ep)/h}$ accuracy. $f$'s had accuracy $e^{-\al/h}$. I need $\al>\ep$ so when take product, I get something smaller than $-S$. 
%$\chi_j'$ extend.
%overlap, 0. Error would not be good enough!

$L$ is the matrix of $\cal L$ in $e_h^{(0)}$.
\begin{align}
L &= \pi^{-\rc 2} he^{-S/h} (A + \te(e^{-\al/h}))\\
A&= [-1,1] + hA_1+\cdots\\
M&= \pi^{-1} h e^{-2S/h} \pa{
\matt1{-1}{-1}1+hA_1 + \cdots + \te(e^{-\al/h})
}\\
\psi(\be) &= \sum \be_n\psi_n\\
\psi_n &= h^{-\rc 4} c_h(h) \one_{E_h} e^{-\ph/h}\\
e^{-t\De\ph}\Psi(\be) &= e^{-tM}\Pi^{(0)} \psi(\be) + (I-\Pi^{(0)}) \psi(\be)\\
&= \psi(e^{-t h e^{-2S/h} A}\be) + \te_{L^2} (e^{-1/Ch}).
\end{align}
Higher dimensions are more complicated in the construction of the quasimodes; we have the Hodge Laplacian $d_\ph^*d_\ph + d_\ph d_\ph^*$. 

%Fokker-Planck.
Ambitious extension: Probabilistic interpretation: suppose you're in higher dimension, analogue but not a gradient flow.Vencil-Friedland theory (?). Create Markov chain where equilibrium points... It's a dynamical system in higher dimensions. 

%for what operators have supersymmetric structure?