\def\filepath{C:/Users/oldhe/Dropbox/Math/templates}
%https://github.com/holdenlee/templates

\input{\filepath/packages_article.tex}
\input{\filepath/theorems_with_boxes.tex}
\input{\filepath/macros.tex}
\input{\filepath/formatting.tex}

\pagestyle{fancy}
\lhead{Generative Adversarial Nets}
\chead{} 
\rhead{Simons Institute, Spring 2017} 
\lfoot{} 
\cfoot{\thepage} 
\rfoot{} 
\renewcommand{\headrulewidth}{.3pt} 
\setlength\voffset{0in}
\setlength\textheight{648pt}

\addbibresource{bib.bib}

\begin{document}
\tableofcontents

%Daniel Roy, Shai Ben-David

We'll discuss:
\begin{enumerate}
\item
Informal problem
\item
Basic algorithm
\item
Some properties
\item
Problems
\begin{enumerate}
\item
Vanishing gradients
\item
Mode collapse
\end{enumerate}
\item
State-of-the-art: Unrolled GANs
\end{enumerate}•

\section{Informal problem}
We have a true distribution $P_{\text{true}} = P_{\text{data}}$ over $\R^{D}$ (e.g. $D=1000$). We want to find a (neural network) function $G_\te:\R^d\to \R^D$, called the \vocab{generator}, such that for $Z$ some randomness, $Z\sim$(some given distribution/noise), $G_\te(Z) \sim P_{\text{model}}(Z|\te)$.

\section{Basic algorithm}

We introduce another neural network $D_\phi:\R^D\to \R$ which tries to determine whether the data point (image) is from the real model or from the generator.

The objective function is
$$
\min_\te \max_\phi
\ub{\ba{
-\rc 2\EE_{x\sim p_{\text{data}}} \ln D_\phi(x) - \rc 2 \EE_z \ln (1-D_\phi(G_\te(z)))}}{V(\phi,\te)}.
$$

The optimal $D$ is
$$
D_{\text{opt}}(x) = \fc{p_{\text{data}}(x)}{p_{\text{data}}(x) + p_{\text{model}}(x|\te)}.
$$

In practice, we have a finite number of samples, so the objective is 
$$
\min_\te \max_\phi
\ba{
-\rc 2\sum_i \ln D_\phi(x_i) - \rc 2 \sum_j \ln (1-D_\phi(G_\te(z_j)))}.
$$
%can you switch?

This isn't a convex problem, so you can't switch min and max.

In practice, do simultaneous gradient descent. Take a step in $\phi$, then $\te$, etc.\footnote{Contrast with alternating minimization: given the current generator, find the best discriminator. This is unstable. 
%solving inner loop...
%optimize wrt current generator.
%empirical could be weird.
}
 But the %discriminator 
generator actually wants to perform well against the max.  In unrolled GANs, do $n$ steps of the discriminator update.

What is the goal? We want $G$ to generate a distribution minimizing the JS divergence
\begin{align}
\min_\te JS(p_{\text{model}}(\cdot |\te)||p_{\text{data}})\\
JS(P||Q) =\rc 2KL(P||M) + \rc2 KL(Q||M)\\
M=\rc2 (P+Q).
\end{align}
You can also minimize $f$-divergences,
$$
D_f (Q||P) = \int f\pa{\dd QP(x)}dP(x).
$$fffff
You can get a lot of different divergences/metrics from different $f$.
Ex. One direction of KL is maximum likelihood.
%distance to square: chi squared

Solving for $\te$ in the minimax against a powerful discriminator would be the same as minimizing the divergence.

Two versions of KL:
\begin{itemize}
\item
$KL(p_{\text{data}}||p_{\text{model}})$ prefers to overgeneralize. 
\item
$KL(p_{\text{model}}||p_{\text{data}})$ focuses on one mode.
\end{itemize}•
You can interpolate these two cases.

The problem of mode collapse is some problem with the training procedure, not the

The theorist way is to look at TV (total variation, $L^1$) distance, corresponding to the ideal discriminator (binary).
%exacting notion of distance.

\section{Problems}
\subsection{Vanishing gradients}
If you try to max in $\phi$ and min in $\te$, if the discriminator gets ahead of the generator, the
 gradients of generator are close to 0, and discriminator gets generator in situation where it doesn't know how to improve itself. So actually use a different objective function.
 
% The trick to get simplify 
For other $f$ divergences, the generator is also minimizing a different objective function. %Modify the generator
 
 %payoff change
2nd term: on fake data, minimize probability of mistake. Instead, minimize
$J^{(G)} = -\rc2 \EE_{z}\ln D(G(z)).$
You tend not to have the vanishing gradient problem.
%can't massage into divergence?

\subsection{Mode collapse}

This is serious. Fitting on mixtures of Gaussians, over time it fits a mode, jumps to another, and meanders around. The generator does well on one mode. The discriminator chases it around; the generator does not converge.
%is there good solution?

Unrolled GANs: see that it focused too much somewhere, and penalize. 
%model the fact that , penalize from not 

Rather than discriminator take a single observation, get a bag of samples from the true and fake data; do a 3-sample test---which is more like the real distribution?
%alternating gradient descent.


%not sgd with momentum, but Adam, inspired in AdaGrad. AdaGrad with extra momentum.

Mode problem is generally a problem with log likelihood: it rewards you for matching a mode.

 Sometimes caps the discriminator to not make strong predictions.                                                                         
% One sided label distribution.
%Leon Bottou noises.

%if two are close to subspace...
%AM not easy to analyze.

%zero-sum: in general expect that.
%try to improve term, not max of that term.

The point of generator model is that you hope that in learning a good generator, you somehow unlock the structure in the data. There are tricks to making the coordinates more interpretable.

\section{Theoretical problem}
%Formal problem}

 
We have a distribution $p_{\text{true}}$. We have a way of representing a distribution, a neural net, and want to get as close as possible.

Suppose we have a family of representations of distributions.
$$
F=\set{G(z)}{z\sim \text{Uniform}, G\in \pat{class of NNs}}.
$$
\begin{itemize}
\item
What is the input? $x_1,\ldots, x_m\sim p_{\text{true}}$.
\item
What do we mean by ``minimize''? We define a distance between distributions $D$.
%Make as close as possible.
\end{itemize}•

I will show this is too hard for any distance, and we need to restrict it to make it feasible. We want to find a mathematical valid formulation that reflects natural assumptions. (Find formalization that explains reality.) 

What is the sample, computational complexity of this problem, ignoring algorithmic issues?

Task: Find $G\in F$ that minimizes $D(p_{\text{true}}, G(z))$.

We introduce the two sample problem. There are two unknown distributions $P_1, P_2$. 

\begin{prb}[Two sample problem]
Given samples $S_1\sim P_1^m$, $S_2\sim P_2^m$, determine: is $P_1=P_2$ or $D(P_1,P_2)>\rc 3=\ep$?
\end{prb}


Consider a simple version. 
\begin{enumerate}
\item
Toy problem 1: 
$P_1=\text{Uniform}[1,\ldots, N]$, $P_2 = \text{Uniform}[1,\ldots, \fc N2]$. Take a constant number $O(1)$ of samples, see if you have any points in $[\fc N2+1,\ldots, N]$. 
\item
Toy problem 2:
Now consider $P_1$ known, $P_2$ unknown.

$P_1=\text{Uniform}[1,\ldots, N]$, $P_2 = \text{Uniform}(A)$ where $A\subeq [1,\ldots, N], |A|=\fc N2$. 

Now $n=\Om(\sqrt N)$. As long as we don't see a point repeating, we cannot tell if it's uniform from the whole subset or a subset. We need $\Om(\sqrt N)$ to see repetitions.
\end{enumerate}•
The key difference between these 2 cases: is the problem symmetric?

%restrict family of discriminators. Ways of restricting. How can this solve the sample complexity.
%SA: not question reality.
%
To make the problem feasible (solve the blowup of sample complexity), I'm going to restrict the discriminator, not the distributions.
I want the sample complexity $m(\ep,\de,d)$ to depend on just the distance $\ep$, and probability of failure $\de$, and the power of discriminator $d$ (ex. they are in a family of bounded VC dimension).
 %The class of discriminators will give our definition of distance.
%Which generator is close to the truth? With finite samples, estimate the distances.

We want to estimate $D(p_{\text{true}}, G_1)$, $D(p_{\text{true}}, G_2)$. %VC dimension.
%Family of discriminators of VC dimension

We pick a notion of distance $D$ based on our class of discriminators such that we have finite sample size approximation guarantees.

Let $A$ be a family of subsets of $X$ (domain of $P$), 
$$
D_A(P_1,P_2) = \sup_{a\in A}|P_1(a)-P_2(a)|.
$$
This looks like total variation distance, except we restrict to $a\in A$.

This is a metric; it can be justified by taking a set of subsets we're interested in seeing differences in. It follows from VC theory that if $A$ has some finite VC-dimension then the probability
$$
\Pj_{S_1\sim P_1^m,S_2\sim P_2^m}
[|D_A(S_1,S_2) - D_A(P_1,P_2)|>\ep]
<O\pf{md}{l^{m\ep^2}}.
$$
You can generalize from sets to functions. Replace the sup by 
$$
\sup_{f\in A} |\E_{P_1}(f) - \E_{P_2}(f)|.
$$
We get pseudodimension, or dimension of a family of functions.

This is actually the type of distance we are trying to minimize!

Ex. subsets are all intervals. $P_1$ is uniform over odd points, $P_2$ is uniform over even points, with respect to intervals they are very similar; no interval has many more/less odd than even points.

%It is 
We assume the difference between distributions is detectable by the family of sets/functions.

We were looking at sensor networks; detect whether there was a real change/event.

%some distribution, tweak. 
Can we come up with a family that captures what humans can distinguish?

%If you cannot distinguish between a good/bad generator.
%can you distinguish between good/bad.

A discriminator is only a NN of certain size. We can talk about sample sizes. This doesn't say anything about the generator.

%iff? 
How general is this formulation of distance? Converse: If there is distance with finite sample size guarantee, does it have this form (or something similar)?

%Gowers norms

%INSERT_HERE
\end{document}