Wiki page: \url{https://simons-institute.github.io/pseudorandomness/groups/modelstructure.html}


A theme that cuts across many domains of computer science and mathematics is to find simple representations of complex mathematical objects such as graphs, functions, or distributions on data. These representations need to capture how the object interacts with a class of tests, and to approximately determine the outcome of these tests.

For example, in machine learning, the object might be a distribution on data points, high dimensional real vectors, and the tests might be half-spaces. The goal would be to learn a simple representation of the data that determines the probability of any half-space or possibly intersections of half spaces. In computational complexity, the object might be a Boolean function or distribution on strings, and the tests are functions of low circuit complexity. In graph theory, the object is a large graph, and the tests are the cuts In the graph; the representation should determine approximately the size of any cut. In additive combinatorics, the object might be a function or distribution over an Abelian group, and the tests might be correlations with linear functions or polynomials.

The focus of the working group is to understand the common elements that underlie results in all of these areas, to use the connections between them to make existential results algorithmic, and to then use algorithmic versions of these results for new purposes. For example, can we use boosting, a technique from supervised learning, in an unsupervised context? Can we characterize the pseudo-entropy of distributions, a concept arising in cryptography? Do the properties of dense graphs ``relativize'' to sub-graphs of expanders?

In particular, we'll start from boosting, a technique in machine learning to go from weak learning to strong learning, i.e., taking an algorithm that learns a function only with a small correlation and making one that learns the function on almost all inputs. We'll show how boosting implies a general Hardcore Distribution Lemma, showing that any function that cannot be $1-\delta$ approximated by simple functions has a sub-distribution of size $\delta$ where it has almost no correlation with simple functions. By starting from boosting, we will be able to show a constructive version of this lemma. From the Hardcore Distribution lemma, we'll derive the Dense Model Theorem used by Green and Tao to show arbitrarily long arithmetic progressions in the primes. Again, by starting with boosting, we get a general algorithmic version of DMT. This algorithmic version can then be used to derive a general Weak Regularity Theorem, with that of Frieze and Kannan and analogs for sparse graphs as a special case.

Hopefully, at this point, the working group will segue from known connections to new connections, e.g., is there a strong boosting that implies strong regularity? Can algorithmic regularity lemmas be used in ML?

We won't assume any background and will develop everything from first principles using only simple calculations. Here's an optional reading list, and some papers we might refer to.

Papers with results we'll cover:

\begin{itemize}
\item
Klivans and Servedio, Boosting and Hard-core Sets, FOCS 99.
\item
Omer Reingold, Luca Trevisan, Madhur Tulsiani, Salil P. Vadhan: Dense Subsets of Pseudorandom Sets. FOCS 2008: 76-85
\item
Luca Trevisan, Madhur Tulsiani, Salil P. Vadhan: Regularity, Boosting, and Efficiently Simulating Every High-Entropy Distribution. IEEE Conference on Computational Complexity 2009: 126-136
\item
Russell Impagliazzo, Algorithmic Dense Model Theorems and Weak Regularity
\item
Sita Gakkhar Russell Impagliazzo Valentine Kabanets. Hardcore Measures, Dense Models and Low Complexity Approximations
\end{itemize}•

Bibliography:

We won't go through these papers explicitly, but they provide the context.
\begin{itemize}
\item
Robert E. Schapire: The Strength of Weak Learnability (Extended Abstract). FOCS 1989: 28-33 : 01 June 2005 A desicion-theoretic generalization of on-line learning and an application to boosting Yoav Freund, Robert E. Schapire
\item
Yoav Freund, Robert E. Schapire: Game Theory, On-Line Prediction and Boosting. COLT 1996: 325-332
\item
Russell Impagliazzo: Hard-Core Distributions for Somewhat Hard Problems. FOCS 1995: 538-545
\item
Thomas Holenstein: Key agreement from weak bit agreement. STOC 2005: 664-673
\item
Boaz Barak, Ronen Shaltiel, Avi Wigderson: Computational Analogues of Entropy. RANDOM-APPROX 2003: 200-215
\item
Alan M. Frieze, Ravi Kannan: The Regularity Lemma and Approximation Schemes for Dense Problems. FOCS 1996: 12-20
\item
Noga Alon, Amin Coja-Oghlan, Hi\^ep H\`an, Mihyun Kang, Vojtech R\''odl, Mathias Schacht: Quasi-Randomness and Algorithmic Regularity for Graphs with General Degree Distributions. SIAM J. Comput. 39(6): 2336-2362(2010)
\item
Noga Alon, Assaf Naor: Approximating the Cut-Norm via Grothendieck's Inequality. SIAM J. Comput. 35(4): 787-803 (2006)
\item
Green, Ben; Tao, Terence (2008). ``The primes contain arbitrarily long arithmetic progressions''. Annals of Mathematics. 167 (2): 481–547.
\item
Tao, Terence; Ziegler, Tamar (2008). ``The primes contain arbitrarily long polynomial progressions''. Acta Mathematica. 201 (2): 213–305
\end{itemize}•
