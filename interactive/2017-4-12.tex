\section{Active classification with comparison queries (Shay Moran)}

We study an extension of active learning in which the learning
algorithm may ask the annotator to compare the distances of two
examples from the boundary of their label-class. For example, in a
recommendation system application (say for restaurants), the annotator
may be asked whether she liked or disliked a specific restaurant (a
label query); or which one of two restaurants did she like more (a
comparison query).

We focus on the class of half spaces, and show that under natural
assumptions, such as large margin or bounded bit-description of the
input examples, it is possible to reveal all the labels of a sample of
size $n$ using approximately $O(\log n)$ queries. This implies an
exponential improvement over classical active learning, where only
label queries are allowed. We complement these results by showing that
if any of these assumptions is removed then, in the worst case,
$\Omega(n)$ queries are required.

Our results follow from a new general framework of active learning
with additional queries. We identify a combinatorial dimension, called
the \emph{inference dimension}, that captures the query complexity
when each additional query is determined by $O(1)$ examples (such as
comparison queries, each of which is determined by the two compared
examples). Our results for half spaces follow by bounding the
inference dimension in the cases discussed above.

If time allows, we will discuss several directions for future research
and some open problems.

Joint work with Daniel Kane, Shachar Lovett, and Jiapeng Zhang

\subsection{Active learning}

In active learning, get unlabeled data, ask for labels of a few data points, and predict as well as in supervised learning. When is this possible?

For example, for thresholds in 1D, do binary search.

Half-planes in 2-D: Given unlabeled points $x_1,\ldots, x_n\in \R^2$. Labels are $y_i = \sgn(\an{w,x}-t)$. Any active algorithm requires $\Om(n)$ queries. Ex. Take points in convex position; each can be separated from the rest by a line.

\subsection{Active learning with additional (comparison) queries}
We allow more interaction. What kind? It doesn't make sense to allow arbitrary additional queries. Asking arbitrary membership queries can be problematic (Lang, Baum).

A relative query is a query that holds relative information. ``A vs. B?'' ``Is A more like B or C?''

Let $H=\set{\sign(f)}{f\in F}$ where $F$ is a set of real functions, e.g. half-spaces and neural-nets. Label-query asks $\sign(f(x_i))$ and comparison query asks $f(x_i)\ge f(x_j)$?



\subsection{Learning half-spaces with comparison queries}
%error if close?

Half-planes in 2D: 
This allows active learning with $O(\log n)$ queries. 
\begin{enumerate}
\item
Sample $S$ of size 20.
\item
Label $S$.
\item
Find points in $S$ closest to line.
\item
Build cones
\item
Label inputs in cones (must be correct)
\item
Repeat on unlabeled points.
\end{enumerate}
With probability $\ge \rc2 $ over samples, each iteration labels half of the remaining unlabeled points. Each iteration makes $O(1)$ queries, and so whp $O(\log n)$ are sufficient.

In $\R^3$ this fails. There are $n$ points in $\R^3$ that require $\Om(n)$ label/comparison queries for revealing all labels.

But we can do better if data has bounded bit complexity, $x_i\in \{0,\ldots, B\}^d$. Then all $n$ labels can be revealed using $\wt O(d\ln B\ln n)$. Note $d\ln B$ is the size in bits of each example.

Theorem: Sample in $\R^d$ with margin $\ga$. All $n$ labels can be revealed using $\wt O(\ln(1/\ga) \ln n)$ label and comparison queries.

Our result is actually with respect to $\fc{\min_i |f(x_i)|}{\max_i |f(x_i)|}$. This scales with $f$. 

%pigeonhole, volume argument
%cf. ellipsoid and vaidya's cutting-plane

Can we get dimension independent bounds? No. There is a sample of $n$ points in $\R^{n+1}$ with margin $\Om(1)$ that requires $\Om(n)$ label/comparison queries for revealing all labels.

\subsection{Inference dimension}

This is the combinatorial parameter we used to analyze.
\begin{df}
The inference dimension of $H$ is the minimal $k$ such that every realizable sample of size $k$ contains a point whose label can be inferred from the comparison and label queries on the remaining points.
\end{df}
\begin{clm}
The inference dimension of thresholds is 3.
\end{clm}
If 3 points have the same labels, the midpoint can be inferred. If 3 points have different labels, an endpoint can be inferred.

Inference dimension of half-planes is at most 7 (in fact 5). 

Must contain subsample of size 4 with same labels. The cone pointed at the nearest point is defined by 3 points. Any other point can be inferred.

Consider class of $H=\set{\sign(p)}{\deg(p)\le d}$, where $p:\R\to \R$ is univariate polynomial. Open exercise: what is the inference dimension of $H$?

%
\begin{thm}
Let $k$ denote the inference dimension of $H$.
The labels of any realizable sample of size n can be  revealed using $O(k\ln k\ln n)$ comparison and label queries. 
\end{thm}
This can be extended to other types of additional queries.

Lower bound: There exists a realizable sample of size $k$ that requires $\Om(k)$ comparison/label queries.

Proof of upper bound: 
\begin{enumerate}
\item
Sample $4k$ points.
\item
Query them. (Label and comparison)
\item
Infer queries among unlabeled points.
\item
Remove labeled points and repeat.
\end{enumerate}•
With probability at least half over sample, each iteration labels half of the remaining unlabeled points. Use boosting.

Inference dimension for bounded bit-complexity: 
\begin{thm}
Consider $(X,H)$, $X=\{0,\ldots, B\}^d$ and $H$ the class of $d$-dimensional half-spaces. 
Inference dimension of $(X,H)$ is $\wt O(d\ln B)$.
\end{thm}
\begin{cor}
There is a $2k$ linear decision tree for $k$-sum of depth $\wt O(n)$. 
\end{cor}
%dobkin lipton 78, ailon chazelle 05
%Matches lower bounds and improves upper bounds.

\begin{thm}
Let $H$ be class of $d$-dimension half-spaces with margin at least $\ga$ wrt $X$.
The inference dimension of $(X,H)$ is $\wt O(d\ln (1/\ga))$.
\end{thm}

Future research:
\begin{itemize}
\item
Relative queries
\item
Restricted membership queries.  (Problem with just membership queries: what if data is on manifold, and query outside manifold?)
\begin{itemize}
\item
Use generative model $g:[0,1]^k$ to generate meaningful membership queries. %(Data augmentation generator.)
\end{itemize}
\item
Streaming model
\item
Noisy comparisons
\item 
Exact learning of threshold function on the boolean cube
\end{itemize}

Consider $(X,H)$, $X=\{0,1\}^d$, $H$ is class of $d$-dimensional half-spaces. Inference dimension is $\wt O(d)$. Thus $2^d$ labels of every $h$ can be revealed using $\wt O(d^2)$ label and comparison queries. This is info-theoretically tight: there are $2^{\Om(d^2)}$ half-planes.

Question: can this be done efficiently?
(The default algorithm is polynomial in space of data, $2^d$.)

On boolean cube, we can find vertex with min value of linear function using linear programming. What about median vertex?
%There exists confident learner that wp $\ge \rc2$ covers at least $\rc2$ of the space.
%$0\le f(x_1)\le \cdots \le f(x_n)$. If you have $x-x_1=\sum_{i<j} \al_{ij} (x_j-x_i)$, $\al_{ij}\ge0$, then $x$ has a positive label.


