\section{Two Paradigms of Semi-Supervised Active Clustering with Sample and Computational Complexity Bounds (Shai Ben-David, University of Waterloo)}

I want to cover three things.
\begin{enumerate}
\item
Introduction/preaching: what should clustering theory do? 
\item
Semi-supervised active clustering (joint work with Hassan Ashtiani and Shrinu Kushagra) \cite{ashtiani2016clustering}
\url{https://arxiv.org/abs/1606.02404}
\item
Learning representations from clustering (Hassan)
\cite{ashtiani2015representation} \url{https://arxiv.org/abs/1506.05900}
\end{enumerate}•

\subsection{Introduction}

Different clustering algorithms yield very different outcomes. Contrast this with other computational tasks, like classification, for which different algorithms roughly give the same results.

Thus the choice of a clustering algorithm is very important. They have different objectives which cannot be satisfied simultaneously. Things we want may include:
%single linkage, 2 means
\begin{enumerate}
\item
Similar points are clustered together
\item
Dissimilar points are in different clusters.
(Already these may conflict---ex. a long chain of points)
\item
Balanced cluster sizes
\item
Stable under perturbation
\end{enumerate}
Where in this simplex of properties do we want?
\begin{itemize}
\item
Single linkage only cares about similar points.
\item
Max linkage (postpone merging far points) only cares about dissimilar points.
\item
$k$-means cares about balancing cluster sizes.
\end{itemize}

Different applications have different priorities. One application is record de-duplication: cluster together records that are the same. Here we would want max linkage: we want dissimilar records not to be identified. For viral spread, we want to group similar points together.

There is no universally optimal clustering.
%at least
There is a need for domain-specific biases.

The first question is what algorithm to use; there is not enough research in this direction. 
%Do I need $k$-means? 
It's not clear what tool is good for that.
Most people use $k$-means without thinking. ``Because everyone else is using this algorithm.''
%crusade, preach.

%We can split into several types: quantitative objectives, problems that you can only tell qualitatively.

We can address the problem %of finding a clustering tool 
in two ways: 
\begin{itemize}
\item
I give similarity of pairs. Find the right clustering algorithm from that.
\item
Here is the tool I'm using, but I will play with the similarity.
\end{itemize}
We focus on the second approach. 
There are ways of asking for input from the user that is more intuitive than asking them to choose the objective---ex. ask them how to cluster a small sample of points.

\subsection{Semi-supervised active clustering}

%This has implications for computational complexity.
See NIPS2016.

Here is the setup.
\begin{itemize}
\item
We have unknown clustering $C_1,\ldots, C_K$ of $(X,d)$ where $d$ is known.
\item
The algorithm can interact with an oracle that knows $C_1,\ldots, C_K$.
\item
The type of interaction is queries of the form are $x_1,x_2$ in the same cluster or different clusters.
\end{itemize}
The algorithm computes, then decides which points to ask about next, etc.

If the user already knows, why do you need the clustering algorithm? There are so many data points; you don't want to ask $n^2$ queries. You need the algorithm to overcome the enormity of the data.
There are specific applications for which this model is relevant, ex. data de-duplication. Here the number of clusters is large compared to the number of records.
%learn a general... 

The number of queries needed is logarithmic in the dataset.

I show the result that under some assumptions on the target clustering, there exists an algorithm that requires $O(k\ln n)$ queries and finds the target clustering in linear time $O(kn(\dim))$, where $\dim$ is the dimension.

Without the queries the task is NP-hard. After log queries, the task collapses to linear time!

We assume that answers are noise-free.

We don't need to know $k$ in advance.
Hardness results kick in for $k$ a function of $n$.

Trivially, $kn$ queries always suffice. Find one representative in every cluster. For every new point, ask whether it's in the same cluster with every representative.

Here are the assumptions.
\begin{enumerate}
\item
Center-based clustering. Every point belongs to the nearest center (Voronoi cell).
\item
$\mu_i$, the center of $C_i$, is $\E_{x\sim C_i} f(x)$, for some known $f$. This definitely holds for $k$-means.
%what's $f$?
\item
Niceness (clusterability) assumption: $\ga$-margin.

\begin{df}
$C_1,\ldots, C_k$ satisfies the $\ga$-margin condition if 
$\forall i, \forall x\in C_i, \forall y\nin C_i$, 
$$
d(x,\mu_i)(1+\ga) \le d(y,\be_i).
$$
\end{df}
Around every cluster there is an empty margin.
%can take y to be closest point in different cluster.
\end{enumerate}
What do we know about $k$-means with such an assumption? 

\begin{thm}[Hardness result]
$k$-means is NP-hard under $\ga$-margin condition as long as $\ga \le 0.84$. (Euclidean distance)
In general, it is NP-hard for $\ga \le 1$. 
\end{thm}
Here $k\sim n^\ep$. The reduction is from set cover.  %not natural, but satisfy conditions.

Although these conditions look strong, it is still hard to do $k$-means under these assumptions.

Positive result: For $\ga>2$, $k$-means without queries is feasible. Use single linkage and use dynamic programming to search over all prunings. %There is almost a sharp threshold.

\footnote{AWigderson: Could you find a clustering that uses $k\ln n$ centers greedily? Probably. 

Find too many clusters and use queries to prune?}

Usually when people find good clustering algorithms, it's with strong assumptions.
The task becomes hard before the condition becomes natural.

The algorithm with queries:
\begin{enumerate}
\item
Ask enough queries to get ``many'' ($N$) points in one cluster. Ask $Nk$ queries.

(Once you have representatives, you can compare, and we can ask which cluster you belong to.)
%random of enough
\item
Pick a cluster with enough points and estimate its center. (This estimate is good by Chernoff.)
The $\ga$ tells me how many points I need in my cluster to be able to tell whether points are in that cluster.
\item
Binary search to find the cluster radius. Ask whether a point of some distance away is in the cluster. Need $\ln n$ queries.
\item
Delete points in this cluster and repeat.
\end{enumerate}•

%info theoretic
%nondeterministic bits in your machine is amount of information you must get.
%look at reduction and see what you get.
%number of nondeterministic bits is $k$ rather than $n$.

%what if nonunique clustering at beginning?

%penalty for opening new cluster.


\subsection{Learning representations from clustering}

This is a different type of interaction. %I can ask the user to classify small subsets.
Hand the user a small subset $S\subeq X$ and ask to get back their desired clustering of $S$.

What can I learn/generalize from this?

Based on this $S_1,\ldots, S_k$, how can we pick a suitable clustering tool for $X$? Note that maybe not all clusters are represented. (Otherwise, we can think of it as a classification problem.)

The idea is that what we want to learn is the metric. Instead of picking between different clustering algorithms, fix the clustering algorithm (regularized $k$-means) and learn the metric.

We fix a family of embeddings of $X$ into some $\R^d$ (a family of kernels over $X$), $F$.

Now we can phrase the problem more precisely: the algorithmic task is to find $f\in F$ such that 
$
A(f(X))|_S
$
is close to $S_1,\ldots, S_k$.
This is a well-defined problem; we can talk about sample and computational complexity.

This is not an ideal solution because it could be that the number of clusters could be much larger. The user can also give clusterings that are inconsistent when given a small vs. large set.

%minimize on distribution?

%It's hard to justify the assumption that the user is minimizing some objective.

For sample complexity analysis of ERM algorithms in the model, we need a notion of distance between clusterings, and a notion of complexity of $F$.

%dimension of family of embeddings.
%PAC-style results.

The distance is
$$
D((C_1,\ldots, C_k),(C_1',\ldots, C_k')) = 
\min_{\pi \in S_k} \rc{|X|}\sumo ik |C_i \triangle C_{\pi(i)}'|.
$$
%could also be number of disagreements $C:X^2\to $. same, different

Now we can phrase as a PAC problem.
%uniform convergence.
What is the sample size $m^{UC}_F(\ep,\de)$ to get within $\ep$ with probability $1-\de$?
This depends on the generalized pseudodimension of $F$. The pseudodimension is defined as follows. For $F$ is a family of functions $X\to \R$, 
$$
p\dim(F) = \max\set{n}{\exists x_1,\ldots, x_n,b_1,\ldots, b_n, \forall \si\in \{0,1\}^n, \exists f\in F, \forall i\le n, \one[f(x_i)\ge b_i] = \si_i
}.
$$
It is the largest set which we can pseudo-shatter.

We have to generalize pseudodimension to vector-valued functions. The generalize is the maximum pseudodimension of all the projections. %to coordinates?

In some common families of kernels, we can calculate the pseudodimension.
%setup, standard PAC tools

We don't have an efficient algorithm because to find the ERM we need to solve regularized $k$-means. How to use users' information to learn clustering on the whole dataset is interesting and this is only a partial answer.
%change of perspective

There's little work on sample complexity of metric learning---how many pairs you need to get information about.

Would some condition on clustering make it easier? Fat-shattering dimension, etc.

%%%%%%%%%%%%%%%%%
%%from PCMI talk
\begin{comment}

This is a position talk. I'm preaching that everything that you're doing is wrong. 

``The purpose of science is to find meaningful simplicity in the midst of disorderly complexity." --Herbert Simon.

This can also serve to describe the goal of clustering.

Clustering is one of the most widely used tool for exploratory data analysis, but we don't understand it.

I will address 2 topics.
\begin{enumerate}
\item
Model (tool) selection: How to choose the best clustering algorithm for your data? How should you set its parameters (e.g., number of clusters)?
\item
Computational complexity of clustering: Discrepancy between theoretical hardness of clustering and practice.
\end{enumerate}

Unlike other common computational tasks, different choices may lead to significantly different clustering outcomes. It's more crucial than model selection in other tasks. Ex. 2-means, single-linkage.

When you have 300-D data and hundreds of thousands of points, you can't tell!

How do we define clustering? Partition the data sets so that
\begin{enumerate}
\item
similar points reside in the same cluster
\item
non-similar points get separated.
\end{enumerate}•
Often these two requirements cannot be simultaneously met. The definition does not determine how to handle such conflicts.

Pick $\al\ll \be$. Is there a clustering function that clusters every pair of points closer than $\al$ and separates every pair farther than $\be$? No.
}

%We can think of any given clustering algorithm as a point in a simplex whose vertices 
Different clustering tools pick different tradeoffs. 
Single linkage focus on similar points being in the same cluster.  Max linkage focuses on dissimilar points not sharing a cluster.

$k$-means balance clusters and avoid having dissimilar points together.

Examples:
\begin{itemize}
\item
De-duplication of records in a database: The emphasis is on separating dissimilar points.
\item
Clustering people for predicting viral spread (disease or rumors): emphasis on clustering similar points together.
\item
Clustering neighborhoods to school districts: Need for balance between sizes of clusters.
\end{itemize}•
Ex. binning by locality-sensitive hashing. %similar points, same cluster

How do users actually pick a tool for their data? Ad-hoc.

Ease of use, freely downloadable software, worked for friend.

Ex. pick a drug based on colors of the package and price.

Fallacies: ``my algorithm outperforms all others." ``This medication is best, regardless of what your sickness is." There's no universally best clustering algorithm.

Check against inherent structure in the data? Sure, if it looks like a collection of well-separated dense rounded clouds.

What about experimental evidence: For every two reasonable clustering algorithms there are data sets in which one performs better than the other.

Two approaches to tool-selection challenge.
\begin{enumerate}
\item
Axiomatic (property-based): formulte properties of clustering functions that allow translating prior knowledge into guidance concerning the choice of suitable clustering functions.
\item
Data-specific expert input: Must-link/can't link pairs of instances (learn from user), sample clustering of small input subsets. Present small clustered subset to user and ask for feedback.
\end{enumerate}

Is it the case that clustering is hard only when it does not matter (CDNM)?

Worst-case complexity is by far the most cited, researched, best understood approach to analyzing the difficulty of computational tasks. Ex. SAT.

Show easy: reduce to SAT and use a SAT solver. Linear programming, neural network training.

%papadimitriou
%400000 biologist not understanding complexity

List requirements on notions of clusterability. (Hardness coming from cases where there is no structure?)
\end{comment}
