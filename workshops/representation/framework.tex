\section{A formal framework for representation learning (Andrej Risteski, Princeton University)}

We will introduce a formal framework for thinking about representation learning, endeavoring to capture its power in settings like semi-supervised
and transfer learning. The framework involves modeling how the data was generated, and thus is related to previous Bayesian notions with some new twists.
 
In simple settings where it is possible to learn representations from unlabeled data, we show: 
\begin{enumerate}
\item
it can greatly reduce the need for labeled data (semisupervised learning)
\item 
it allows solving classification tasks when previous notions such as nearest neighbors or manifold learning either don't work or require too much data.
\end{enumerate}•
 
We also clarify two important settings---linear mixture models and loglinear models---where representation learning can be done under plausible assumptions (despite being NP-hard in the worst case). 
 
Joint work with Sanjeev Arora. 

I will 
\begin{enumerate}
\item
introduce formal framework
\item
clarify in two settings: mixture and log-linear models
\item
illustrate power of representation learning vs. nearest neighbor classifiers and reducing labeled data.
\end{enumerate}•


Why learn representations? What do we want out of our model? Our approach is motivated by variety of uses and successes. 
\begin{itemize}
\item
Unsupervised learning: discover structure
\item
Transfer learning: features for one task helps for others
\item
Semi-supervised learning: help do classification with smaller number of labeled samples.
\end{itemize}•
I'll focus on the last. Our model is inspired by Bayesian considerations.

\subsection{Formal model}

Think of representation as high-level succinct representation of object. Map is intuitively many-to-one. Our proposal:
$$x=g(h,r)$$
%german short-haired pointer.
where $x$ is raw data, $h$ is representation, and $r$ is random seed. 
The distribution of $g(h,\cdot)$ is raw data manifestions of some representation.

We want benefit from passing to the representation: classification is easier in representation space. There is a classifer from representation space with low error.
%There exists $C$ with low error

Since $x=g(h,r)$, learning a representation is ``inverting'' $g$.

\begin{df}
A function $f$ is $(\be,\ga)$-valid encoder if wp $\ge \be$, 
$$
\ve{f(x)-h} \le (1-\ga )\ve{h}
$$
\end{df}
So for $\al$-Lipschitz classifier $C$, wp $\ge \be$,
$$
\ve{C(f(x))-C(h)}_\iy\le (1-\ga)\al\ve{h}.
$$

This notion depends on what you're trying to classify. This is tied to a task.
%A representation can have many classifiers.
%you need a good classifers
%set for representation space.

Comparison with related notions:
\begin{itemize}
\item
manifold learning: assume points lie on low-dim manifold. Sample complexity is exponential in dimension of manifold.
\item
kernel learning: 
We exhibit encoder examples involving thresholding; it is unclear if they correspond to a kernel.
%linear map, then threshold.
\end{itemize}
Point: our notion is different.

Relationship to Bayesian modeling. 
From Bayesian perspective, we have posterior distribution over $h$. 
\begin{thm}
If $(1-\de^2, \ga)$-encoder exists, wp $1-\de$ over choice of $x$, posterior $h|x$ concentrates close to $f$
$$
\Pj_{h|x} [\ve{f(x)-h}\le (1-\ga)\ve{h}]\ge 1-\de.
$$
\end{thm}
We can exhibit simple settings involving mixture models where 
\begin{itemize}
\item
encoders are efficiently learnable.
\item
encoders help for semi-supervised learning: nearest neighbors in representation space require few samples.
\end{itemize}•

\subsection{Recommendation systems}

See many users and subset of movies they like. There isn't a numerical values. Then ask whether the user likes another movie.

We need to impose latent structure. Make a linear structure (mixture model). There are a bunch of genres, subsets of movies. A  user is mixture of genres. 
$$
\Pj(\text{movie $i$ is emitted}) \propto (Ah)_i
$$
where $A$ is genre matrix and $h$ is genre proportions. Encoder maps liked movies to latent genres $h$, and semi-supervised learning learns encoder first and then does classification knowing $h$.

\begin{thm}[Corollary of AGKM16]
If $A$ is $l_1$ well-conditioned ($\ve{A(h_1-h_2)}_1\succsim \ve{h_1-h_2}_\iy$), then there is good encoder learnable in time $\poly\pat{movies}$ even if number of emitted movies is logarithmic in total number of movies.
\end{thm}
%existence of encoder. Nothing to do with classification tasks.
Classification tasks are the ones that use $h$.
%unique way to reconstruct?
%want something close to correct $h$.
%repeat many times...

\subsection{Power from representations}
We get power from these representations. Quantify by 2 phenomena:
\begin{enumerate}
\item
Nearest neighbors: if $T\ll \sqrt m$, NN doesn't reflect latent genres. If $T\gg \sqrt m$ with natural genre assumptions NN works.
\item
Semi-supervised learning: for constant number of genres, label $ l(h)=\sign(\an{w,2h-\one})$,...
\end{enumerate}
Similar results for log-linear models. 

