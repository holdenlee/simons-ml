\section{How to escape saddle points efficiently (Praneeth Netrapalli, Microsoft Research India)}

We show that a perturbed form of gradient descent converges to a second-order stationary point in a number iterations which depends only poly-logarithmically on dimension (i.e., it is almost ``dimension-free''). The convergence rate of this procedure matches the well-known convergence rate of gradient descent to first-order stationary points, up to log factors. When all saddle points are non-degenerate, all second-order stationary points are local minima, and our result thus shows that perturbed gradient descent can escape saddle points almost for free.

Gradient descent for $\min_x f(x)$: for $l$-smooth functions, converges to points where gradient is close to 0:
$\ve{\nb f(x_t)}<\ep$ in $t=O\pf{l(f(x_0)-f^*)}{\ep^2}$. Convertes to $\ep$-first order stationary points. They can be local minima or saddle points. 

They can converge to either but these are very different in practice. Local minima are often good, but saddle points are very poor compared to global minima. 

Local minima are much more desirable to converge to than saddle points. Gradient descent can converge to saddle points. How to escape saddle points efficiently?

Ge2015: by adding noise, gradient descent escapes saddle points. This requires $\poly(d)$ iterations. Gradient descent comes close, after $\poly(d)$ iterations it escapes. 

Can we do this more efficiently?

Hessian Lipschitz: $\ve{\nb^2f(x)-\nb^2f(y)}\le \rh\ve{x-y}$. 
2nd order stationary point: $\ve{\nb f(x)}\le \ep$ and $\la_{\min}(\nb^2 f(x))\ge -\sqrt{\rh\ep}$. 
(Nesterov Polyak 2006)

Perturbed gradient descent finds $\ep$-second order stationary point in $t=\wt O\pf{\ell (f(x_0) - f^*)}{\ep^2}$. It converts to second order stationary point in essentially same amount of time as GD to first order stationary point.

Keep doing GD, and once in a while it does perturbation (when perturbation condition holds). It kicks in when gradient at current point is small and we haven't added noise for a while. 

Proof idea: two cases.
\begin{enumerate}
\item
$\ve{\nb f(x_t)}>\ep$. Then descends.
\item
$\ve{\nb f(x_l)}\le \ep$. There exists a descent direction.

Let $S$ be set of points around saddle point from where gradient descent does not escape saddle point. Key technical result: $\Vol(S)$ is small. 

Consider adding a negative eigendirection to get $w$. If $u$ does not escape, $w$ escapes. For every point there are many other points that escape. (Divide in lines; it intersects set in small segment.)
\end{enumerate}•
Gradient descent converges to first order stationary points, perturbed GD converges to second order stationary points, depends only logarithmically on dimension.

Further results using local structure:
\begin{itemize}
\item
strict saddle property: every saddle point has strictly negative eigenvalue. Then all second-order stationary points are local minima.
\item
Local strong convexity. Get local geometric convergence once we are in neighborhood.
\end{itemize}•

Open:
\begin{itemize}
\item
is randomness in beginning sufficient?
\item
Do momentum methods help accelerate for nonconvex problems?
%Adam, etc. 
\item
Extensions to stochastic case.
\end{itemize}•