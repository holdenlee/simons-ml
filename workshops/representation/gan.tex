\section{Generalization and Equilibrium in Generative Adversarial Nets (GANs) (Sanjeev Arora, Princeton University)}

This paper makes progress on several open the- oretical issues related to Generative Adversarial Networks. A definition is provided for what it means for the training to generalize, and it is shown that generalization is not guaranteed for the popular distances between distributions such as Jensen-Shannon or Wasserstein. We introduce a new metric called neural net distance for which generalization does occur. We also show that an approximate pure equilibrium in the 2- player game exists for a natural training objective (Wasserstein). Showing such a result has been an open problem (for any training objective).

Finally, the above theoretical ideas lead us to pro- pose a new training protocol, MIX+GAN, which can be combined with any existing method. We present experiments showing that it stabilizes and improves some existing methods.

Joint work with Rong Ge, Yingyu Liang, Tengyu Ma, Yi Zhang. 

There hasn't been much theoretical analysis of GANs.

Take Gaussian noise, input through neural net (denoising autoencoder, variational autoencoder) and transform into image. The hope is that the process teaches us something about the structure of images. 
There is concrete evidence in MNIST results, etc.

%
%Do modes correspond to different faces?


Prologue: In 2013 I visited Google and talked to Geoff. I asked Geoff, ``I understand anthropomorphic principle that vision is represented by neural net. Why do you think realistic distrbutions are expressible by a small shallow deep net?'' ``Neural nets are like universal basis that can approximate almost anything very efficiently.'' Cf. tarp drapes over any object.

Geoff's ``neural net hypothesis.''

This is inconsistent with curse of dimensionality. In $d$ dimensions there are $\exp(d)$ directions whose pairwise angle is $>60^{\circ}$.

The number of distinct distributions is $>\exp(\exp(d))$ by discretizing. 
Counting arguments shows we will need neural nets of size $\exp(d)$ to represent some of these distributions.

I thought: Real-life distributions must be special in some way... %Real-life distributions don't come from neural nets

For the last 5 years, I though that you have to make assumptions about distributions; in this work I won't.

We're trying to train a generative model to convert noise into faces. Train by training a discriminator net $D_v$, to train between realistic and synthetic images. This is reminescent of pseudorandomness. %Distinguish between real 

Objective is 
$$
\min_{u\in U}\max_{v\in V} \EE_{x\sim D_{real}} [D_v(x)] - \EE_h [D_v(G_u(h))],
$$
Wasserstein GAN (Arjovsky et al. 17). 

Repeat until convergence: backprop update on discriminator. %to say 1 on real inputs and 0 on 

%no significance to gaussian noise. 
Think of GAN as 2-player game. Most theorists like the idea of GAN because it presses the  right buttons: pseudorandomness, game theory.
Generator plays a net, discriminator plays a net.

When to stop? A necessary stopping condition is equilibrium: payoff unchanged if we flip the order of moves.

In practice this is very unstable and it's believed there's no equilibrium.
\begin{enumerate}
\item
I'll address generalization: suppose generator has won and discriminator has been left with no option but random guessing. Does this mean in some sense that the true distribution has been learnt?

Past analyses: If discriminator capacity and number of samples are very large, then yes because Wasserstein distance is exactly the $\max_{v\in V} \EE_{x\sim D_{\real}} [D_v(x)] - \EE_h[D_v(G_u(h))]$.
%disc pay optimally
\item
Equilibrium: does equilibrium exist in this 2-person game? A priori not guaranteed, ex. rock/paper/scissors.
You can convert this to a GAN example.
\end{enumerate}•

Bad news: bounded capacity discriminators (with $n$ parameters) are weak.  
Suppose I take a uniform distribution on $\fc{n\ln n}{\ep^2}$ random samples from $D_{real}$. 

\begin{thm}
If generator has capacity $n$ its distinguishing probability between these two distributions is $<\ep$.
\end{thm}
Proof: $\ep$-net argument.

\begin{enumerate}
\item
Still holds if many more samples available from $D_{real}$, including any number of held-out samples.
\item
Suggests current GAN objectives may be unable to enforce sufficient diversity in generator's distribution.
\end{enumerate}
%Capacity of discriminator and generator.

Hope internal parameters give insight into images. 

Small discriminator cannot distinguish between low and high entropy.

Take 1000 samples; you see repeated faces, this suggests by birthday paradox diversity is not high.

Partial good news: if number of samples is $>\fc{n\ln n}{\ep^2}$ then performance on samples tracks (within $\ep$) performance on full distribution. Generalization does happen with respect to ``neural net distance'',
$$
\max_{D\in U} \ab{\E_{x\in D_{real}}[D(x)] - \EE_{x\in D_{synth}}[D(x)]}.
$$
(Similar theorems in pseudorandomness.)

Generalization happens for NN distance. 
\begin{enumerate}
\item
Deep nets are Lipschitz with respect to trainable parameters: changing parameters by $\de$ changes output by $<C\de$ for some small $C$.
%only appears as log in number of samples
\item
If number of paramteres is $n$, there are only $\exp(n/\ep)$ fundamentally distinct deep nets (others are $\ep$-close).
\item
For any fixed discriminator, once we draw $>\ln n/\ep^2$ samples, then with probability $\le \exp(-n/\ep)$ distincuishing ability is not within $\ep$ of distinguishing ability for full distribution.
\end{enumerate}•
By union bound, empirical NN distance on $n\ln n/\ep^2$ samples tracks overall NN distance.

Suppose generator just won. If number of samples is somewhat more than trainable parameters of discriminator, then generator would win against all discriminators of full distribution.

But why should generator win in the first place?

Just assume backprop gives optimal discriminator.

%works for other measuring functions
Equilibrium means $D$ gets max paoff from $G$ among all discriminators in class, $G$ ensures min payoff to $D$ among all generators in class. Pure equilibrium may not exist, but we're hoping for it to exist with payoff $=0$.

Thought experiment: instead of single generator net, what if we allow infinite mixture of generator nets? 

Fact: these can represent $D_{real}$ closely (ex. kernel density estimation). %discrete elements tiling space.

Generalization result does not use capacity of generator.

What about finite mixtures of generator nets?

\begin{thm}
A mixture of $n\ln n/\ep^2$ generator nets can produce $D_{synth}$ that looks like (within $\ep$) $D_{real}$ to every deep net discriminator with $n$ trainable parameters. 
\end{thm}

von Neumann min-max theorem: There exists equilibrium if replace discriminator/generator by infinite mixture of discriminators/generators. By our recent observations, in such an equilibrium for the GAN game, generator wins (with payoff $=0$). %solution would win the game.

There is a finite mixture which obtains $\ep$-approximate equilibrium, $D$ get $\ge V-\ep$ and $G$ gets $\le V+\ep$, where $V$ is the payoff in von Neumann equilibrium. 
There exists $\ep$-approximate equilibrium when we allow mixtures of size $n\ln n/\ep^2$. 

Summary: Infinite mixture is powerful, finite mixture is roughly similar, lose only $\ep$.

Existence of approximate pure equilibrium (for Wasserstein objective): Take small mixture (with weights $w_1,\ldots$) approximate equilibrium. Show it can be simulated by a single deep net. Use some bits in white noise to implement selector circuit to choose generator.
The number of parameters is $\wt O(n^2)$. 

Empirics: The MIX+GAN protocol can be used to enhance GAN game for any existing architecture. Take any GAN code, apply sauce (mixture) on top. Player 1/2 is mixture of $k$ discriminators/generators, ex. $k=3$ to 5.  Maintain separate weights for each component, update via backprop. Use entropy regularizer on weights to discourage collapsing.

%disc could be just 1.
Even with $k=3$ quality is better. (CeleA Faces Dataset) Adding capacity is not trivial. %discriminator . 
\footnote{Generator could be nonparametric. Take and add noise to it. It doesn't train, it's pretrained with data, so in a way has infinite capacity, gets close to kernel approximation.}

%More memorizing?
Memorization is not that easy. It's easier to memorize with a mixture of networks. Are you pushing towards  memorization with more networks. Will backprop find it?

Inception score: epoch by epoch MIX+DCGAN does better (Inception score). It's not trivial to get improvements like this. I'm not convinced Inception score is a good measure. 

Also do MIX+WGAN. We get much better objective.

Epilogue: 
\begin{itemize}
\item
Infinite mix of very simple generators closely represent any $D_{real}$.
\item
Reasonable size subsample of generators produce $D_{synth}$ indistinguishable from $D_{real}$ by any small net.
\item
$D_{synth}$ should look like $D_{real}$ to us if our visual system is small neural net.
\end{itemize}•

%Mechanism enforce better generator?

LSH, look at WGAN output with nearest neighbor?

Give generator a bit more capacity, generator can win by sampling uniformly. That happens because it has more capacity. Generator capacity has to limited---how? %What happens


