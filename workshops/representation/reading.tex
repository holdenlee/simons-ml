\section{Representation learning for reading comprehension (Russ Salakhutdinov, Carnegie Mellon University)}

In this talk I will focus on discussing deep learning models that can find semantically meaningful representations of words, learn to read documents and answer questions about their content. First, I will introduce the Gated-Attention (GA) Reader model, that  integrates a multi-hop architecture with a novel attention mechanism, which is based on multiplicative interactions between the query embedding and the intermediate states of a recurrent neural network document reader. This enables the reader to build query-specific representations of tokens in the document for accurate answer selection. Second, I will next introduce a two-step learning system to question answering from unstructured text, consisting of a retrieval step and a reading comprehension step. Finally, I will discuss a fine-grained gating mechanism to dynamically combine word-level and character-level representations based on properties of the words. I will show that on several tasks, these models significantly improve upon many of the existing techniques.

Joint work with with Bhuwan Dhingra, Zhilin Yang, Yusuke Watanabe, Hanxiao Liu, Ye Yuan, Junjie Hu, and William W. Cohen

I'll show a trick to train LSTM's more efficiently.

\subsection{Multiplicative and fine-grained attention}

We have a who-did-what dataset. Most existing systems are far from human-level performance. 

The cloze-style Q/A task involves tuples $(d,q,a)$ where $d$ is document, $q$ is question over contents, and $a$ is answer. The answer comes from fixed vocab $A$. 

Task: given $(d,q)$, find $a\in A$ answering $q$.

Half of work is from DeepMind. A lot of models rely on recurrent networks, 
$$h_t=\phi(Uh_{t-1}+W_t+b),$$
where $h_t$ is hidden state at time $t$, $x_t$ is input at time $t$, and $\phi$ is nonlinearity.

We replace with elementwise multiplication
$$
\phi(Uh \odot Wx + b)
$$
or 
$$
\phi(\al\odot Uh \odot Wx + \be_1\odot Uh + \be_2\odot Wx+b).
$$
This is easer to optimize.
%vs. inductive bias?

If we train regular RNN, we get saturation effect. When you apply $\tanh$, they go to sides, and it becomes hard to optimize. For the multiplicative model, it sits in the middle, in the linear regime. We have much faster convergence.

%Do you have sparsity? Weights don't go to 0, but a lot sit around 0.

If I take a derivative with respect to $W$ in the multiplicative system, $x$ can have direct effect. (Hidden state can affect gradient.) The way gradients are flowing. Gating has big impact on performance.

%This is a diagonal approximation. Could 

%Why use tanh? 

%glove vs. word2vec
Off-the-shelf GloVe words better than word2vec.

Use bi-directional LSTM/GRUs %incopr a little context around
to encode both document and query. 
\begin{itemize}
\item
Use attention mechanism, 
\begin{align}
\al_i &=\text{softmax}(Q^Td_i)\\
\wt q_i &= Q\al_i\\
x_i &= d_i \odot \wt q_i.
\end{align}
Use elementwise multiplication operator to model interactions between $d_i$ and $\wt q_i$. 
\item
Multi-hop architecture. 
\item
Create a softmax. Aggregation happens at output.
$$ P(c|d,q)\propto \sum_{i\in \mathbb I(c,d)}s_i.$$
\end{itemize}
There's so much work done in this space.  Gating mechanisms do reasonably well. The pace is very fast.

When we look at attention models, we sometimes see interesting things. On the $x$-axis is the query, on the $y$-axis is the document. 
%``alleged corruption 

%alleged corruption -> Rod Blagojevich.
%Ex. Senate -> Blag

Words vs. characters:
\begin{itemize}
\item
Word-level representations are good at learning the semantics of the tokens.
\item
Character-level representations are more suitable for modeling sub-word morphologies (cat, cats). 
\item
Hybrid word-character models have been successful. A commonly used method is to concatenate these two representations. 
\end{itemize}

Fine-grained gating:
$$
h = g\odot c + (1-g)\odot (Ew).
$$
Gating happens as elementwise multiplication. If we use scalars it doesn't work as well.
Here
$$
g = \si(W_gv + b_g). 
$$
%sigmoid in (0,1)
We use a bunch of prior knowledge to decide whether to use word or character.

High gate values correspond to character-level representations. Low gate values mean word-level representations. Low-frequency words and proper nouns use character-level representations.


\subsection{Incorporating knowledge as explicit memory for RNN's}


LAMBADA dataset: answer who. 
Coreference resolution is the standard technique. People in NLP are familiar with this. A lot of existing models don't take this into account. 

%Can we incorporate
Can we learn from data and incorporate coreference dependency parse, etc. other NLP info that's useful.

Introduce coreference links, hyper/hyponyny links like skip-connections. This is like having  multiple RNN chains with skip-connections. Construct parse matrix. Construct in such a way so it's a RNN model with additional links.
The directionality matters. Move forward and backwards. We can take the question into account. 

Mathematically, putting a link means that at every hidden state, incorporate info from all links coming in.

%What are parts that
%What weights are shared? Think of this as skip-connections.

It's an interesting and bad results. For babiQA, go from 75.2\% to 91.3\%. For others, we don't get significant gains.
%lambada, cnn
%lambada prediction
%others: attending

Q: how many objects is Sandra carrying?

GA model defaults to 1. Adding structure helps.

\subsection{Generating text} %ve adversarial nets

Leverage unlabeled text for question answering.

It's easy to find an answer. Use POS/NER/parsing to extract possible answer chunks. Anything can be answer. Assume answer is available. 

Given an answer, can you generate a question for it?

We have labeled data $(p,q,a)$ and unlabeled data $p,a$. Generate with seq2seq with copy mechanism. Maybe we can combin to train QA model. One piece that's difficult is to generate the question.

Some simple baselines work well. 
\begin{enumerate}
\item
Given answer, look at some surrounding context. This works remarkably well (in low data regime). 
\item
GAN: One objective function differentiates between true or fake question, the other tries to answer.

One problem is that it's hard to generate good-quality questions, and harder to discriminate between real and fake.
\end{enumerate}•
We do something different, generative domain-adaptive nets. 

Have a data tag that says whether data is coming from true distribution. %Train generator to generate whatever sentence. %so that if I pass through discriminator.
Objective is not to generate coherent sentences, but whethever helps answer the questions. 

We see gains in the low data regime. In high data regime, we still see edge but gap closes. 

We started thinking about: is there a way to generate better looking questions? 

We looked at VAE's. We have $z\sim N(0,I)$ and a code $c\sim \text{Bernoulli}(\mu)$.

We can fix $z$ and change code to change sentiment, tense, etc.

We can fix the code and generate the latent state. Generate a sentence which is negative and past, etc.
(Some sentences are new.) 
How do you evaluate quality? There's no good answer.

I can flip code and regenerate sentence: take a negative sentence and make it positive, etc.

%When you have pointwise multiplication and addition, you have something close to a field. (???)
%powerful computational mechanism
%work in linear regime
%boost up dimensionality, may not have to do gradient descent.

%locality: look at things around
Who-did-what dataset: humans cannot answer the question by  looking at last sentence. Our model does better on these types of questions. 