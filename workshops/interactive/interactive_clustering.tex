\section{Interactive clustering (Pranjal Awasthi, Rutgers University)}

Abstract: Clustering is typically studied in the unsupervised learning setting. But in many applications, such as personalized recommendations, one cannot reach the optimal clustering without interacting with the end user. In this talk, I will describe a recent framework for interactive clustering with human in the loop. The algorithm can interact with the human in stages and receive limited, potentially noisy feedback to improve the clustering. I will present our preliminary results in this model and mention open questions.

Most theoretical work on interaction is in supervised learning (PAC learning, active learning, membership queries, equivalence queries). This work looks at interaction  for unsupervised learning. Can interaction be useful?

In clustering, given $n$ objects, we want to partition them into $k$ disjoint clusters. We want to design provably good algorithms. 

The classical approach is to define an objective function like $k$-means
$$
\sum_i \min_j \ve{x_i-c_j}^2.
$$
This requires the objective function to actually be minimized with the actual clustering.

Another approach is to use a generative modeling approach: data is generated from a mixture of Gaussians, stochastic block models, etc. We have optimal recovery under these assumptions, but this is not robust; we can't say much if the modeling assumption is wrong.

Both approaches aim to remove ambiguity from the problem. Can we remove ambiguity with a human-in-the-loop?
In many scenarios, the best clustering is ambiguous unless the end user is involved, ex. personalized recommendations.

Approaches to interactive clustering include:
\begin{enumerate}
\item
constrained $k$-means/median (constrained given by users) (cf. semisupervised learning)
\item
pairwise clustering  (asking the user whether pairs are in the same cluster or not)
\end{enumerate}
We give a different model.
\begin{itemize}
\item
What can we do with minimal feedback?
\item
Can the model be made noise robust? 
\item
Is the model practical?
\end{itemize}•

Interactive clustering was introduced by Balcan and Blum 2008. This is inspired by equivalence query model for learning. There is an algorithm and teacher. The algorithm proposes function $f_1$, and gets a counterexample $x$ such that $h_1(x)\ne h^*(x)$. Repeat until the algorithm proposes $h^*$.

We do this in the clustering setting. The algorithm proposes clustering $C_1,\ldots, C_k$, gets limited/coarse feedback, and repeats until it proposes the true clustering.
%how to do if agnostic?

We want as few rounds of interaction as possible.

To specify this model, I need to say what the right feedback is. It should be
\begin{enumerate}
\item
easy to provide form teacher's point of view
\item
be informative
\end{enumerate}•
The two kinds of feedback are 
\begin{enumerate}
\item
split: Split$(C_i)$ if $C_i$ contains points from two or more clusters of the ground truth. The teacher does not have to say why.
%why not say?
%we're interested in coarse feedback.
\item
merge: Merge$(C_i,C_j)$ if $C_i$ and $C_j$ are subsets of the same cluster in the ground truth. (In particular they are pure.)
\end{enumerate}
%SD: Must link/cannot link. These would automatically generate constraints of this form? 
Note that ``cannot link'' can be interpreted as a split, but ``must link'' cannot be interpreted here.

The goal is to design an efficient algorithm that attains ground truth in small number of rounds.
\begin{itemize}
\item
Given $n$ points, we can cluster them in at most $n-k$ queries: start with each point in own cluster and merge points when merge feedback is issued.

The teacher is just stepwise saying what the clustering is.
\item
Without further assumptions this is best possible.
\item
What if the clusters are nice?
%\begin{itemize}
%\item
%
%\end{itemize}•
\end{itemize}•
Suppose that membership in cluster $C_i^*$ can be decided by a boolean function $f_i()$, $f_i\in H$. For example, $H$ can be linear separators, rectangles...

\begin{thm}
Suppose $f_i$'s belong to class $H$. Then we can cluster using $O(kd\ln n)$ queries where $d=VCdim(H)$.
\end{thm}
The algorithm is similar to version space algorithms: The version space is all valid clusterings.
\begin{enumerate}
\item
 Start with $VS=\{\text{all valid clusterings}\}$. 
\item
Choose a clustering from VS and present it to the teacher.
Choose by:
\begin{enumerate}
\item
Find a maximal set of points $S$ such that more than half of clusterings in VS put S in a single cluster.
\item
Output $S$ as a cluster. Repeat until points are clustered.
\end{enumerate}•
\item
On feedback, delete all clusterings from VS that are inconsistent with feedback. 
\end{enumerate}•
%how is it when geometry?

This works for every concept class. We can get efficient algorithms for specific function classes.
%
The algorithm does not know $k$; it could generate more clusters.

Claim: After each request, we get rid of at least half of VS. 
\begin{itemize}
\item
On Split$(C_i)$, delete all clusterings that put $C_i$ in a single cluster. 
\item
On Merge$(C_i,C_j)$, delete all clusterings that do not put $C_i\cup C_j$ in a single cluster. (This must be more than half, otherwise $C_i\cup C_j$ is the maximal set.)
%
\end{itemize}•
In practice, show random subsets of each cluster.
%boundary.

Another model: the ground truth $C_1^*,\ldots, C_k^*$ is stable. Assume average stability, which says that on average, points prefer their own clusters over other: %$A$ prefers its own cluster over $A'$. 
For $i\ne j$, $A\subeq C_i^*, A'\subeq C_j^*$.
$$
S_{avg}(A,C_i^*\bs A)\ge S_{avg}(A,A')
$$
\begin{thm}
If average stability holds, $k$ queries suffice.
\end{thm}
Algorithm:
\begin{enumerate}
\item
Construct average linkage tree $T_{avg}$. This is a minimum spanning tree using average distance (to merge children).
\item
Present the root node as a single cluster.
\item
On split feedback, replace cluster with children in $T_{avg}$.
\end{enumerate}•
Claim: $C^*$ is laminar with respect to $T_{avg}$: each cluster is present as a subtree in $T_{avg}$. Otherwise average stability is violated. The teacher will never issue merge requests.

These results are nice but don't lead to practical algorithms.
\begin{itemize}
\item
They don't account for noise in human resposne.
\item
They will present very bad clusterings to the user in order to make quick progress.

We want to show clusterings in a smooth fashion.
\end{itemize}
%Model noise as follows.
Change the model:
\begin{itemize}
\item
$\eta$-noisy merges: If the 
user asks to merge $C_i,C_j$, then $\ge 1-\eta$ fraction of both the clusters belong together in the ground truth.
\item
Local algorithm: The algorithm cannot choose the initial clustering. After a round of feedback, the algorithm can onlhy modify the clusters involved in the feedback. This gives better user experience.

This is useful if you already have a system in place, in use.
\end{itemize}•
Measure progress in terms of how far it is from the actual clustering.
\begin{thm}
If $C^*$ is stable, then we can find it using $O\pa{\fc{\ep_0+k}{1-\eta}\ln n}$-queries where $\ep_0$ is the error of the initial clustering. 
\end{thm}
Algorithm:
\begin{enumerate}
\item
Construct average linkage tree $T_{avg}$.
\item
On split$(C_i)$, find the first node where $C_i$ gets split and go to its children.
\item
On merge$(C_i,C_j)$, find the deepest node containing enough points from $C_i,C_j$ and carve out the pure portion from this node.
\end{enumerate}•
%split is not as direct. does not need to be an actual node.
Can we deal with extremely high noise? 
\begin{itemize}
\item
Unrestricted merges: if user asks to merge $C_i,C_j$ then at least one point from both the clusters belongs together in ground truth.
\end{itemize}•
\begin{thm}
If $C^*$ is stable and request is chosen randomly, then can find it using $O((\ep_0+k)^2\ln n)$ rounds with high probability.
\end{thm}

We chose image segmentation. This is not really the right problem, but it's easy to test. We posed this as a problem on Mechanical Turk. The user can point to regions to merge or split.
This model has the potential to lead to good algorithms.

There are many technical problems:
\begin{itemize}
\item
Algorithm is not scalable, requires average linkage tree.
\item
We need strong stability assumption to prove theorems.
\item
In other applications, presenting the entire clustering is not practical.
\item
What if user is willing to provide more feedback?
\end{itemize}•

Main message: interaction can be useful/critical for unsupervised learning.

%Algorithm gives you optimal $k$ in own sense. Have you compared with cross-validation methods?

QA:

After each step, ask user: Are you happy with segmentation, or do you want to provide more feedback. Some workers were trying to fine-tune boundaries.

Show the same picture to multiple people: %Same phenomenon. 
Did they come to the same answer? In a global sense, they come to the same answer. On boundaries there is some disagreement. In crowdsourcing scenario, take a vote?

Your algorithm wants clusters with respect to underlying similarity or distance. Are users happy with the metric itself? Is there a way to tell your algorithm that?




