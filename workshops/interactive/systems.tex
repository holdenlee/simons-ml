\section{Systems presentations}

\subsection{A deployable decision service (Alekh Agarwal, Microsoft Research New York)}

This is about a contextual bandit system.

Contextual bandit:
\begin{enumerate}
\item
observe state of world (context)
\item
choose action
\item
obtain feedback on chosen action
\end{enumerate}

This is used in MSN.

Can we have a single system that addresses many problems? 

Feedback is only on actions taken. We need to try every plausibly good action for every context.

When doing randomization, record the probability distribution of choices. 

Questions of interest
\begin{itemize}
\item
on-policy: given problem instance, how to quickly learn a good policy while incurring little regret
\item
off-policy: How would have some other policy done if I had used it? Ex. inverse propensity scoring,
$$
V(\pi) = \rc n\sum_{(x,a,p,r)} \fc{rl(\pi(x)=a)}{p}
$$
\end{itemize}

This fails:
\begin{itemize}
\item
separate teams for each part of process: deploy, explore, log, learn.
\item
faulty logs. People drop the probabilities. They don't see why you need the distributions.
\item
runtime behavior incompatible with ML: Ex. business logic overriding randomization---promotional content in first position.
\end{itemize}
There are many subtle errors that are difficult to find in complex systems.

App communicates with client library or web API. Send contexts, decisions, and rewards to join server. Do online learning, changing the model every 5 minutes. (This can respond to breaking news, etc.) Then snde back to client.

Join server also sends training data to data store. Do offline learning, and send hyperparameter features back to online learning.

For evaluation, given a policy, estimate its performance if deployed.

%For optimization, frind a good 
We use $\ep$-greedy exploration. Create a topic model for every document. Context: featurize history.

\url{http://aka.ms/mwt}, \url{http://arxiv.org/abs/1606.03966}, \url{http://hunch.net/~vw}.

\subsection{NEXT: Active learning in the wild (Lalit Jain, University of Michigan)}

Challenge of implementing active learning the the real world: we have recruited participants and our coded up algorithms. In between we need
\begin{enumerate}
\item
GUI for participants
\item
experiment monitoring
\item
traffic monitoring
\item
database
\item
computational backend
\end{enumerate}
Your algorithm is just the beginning; you need a real-time computation system!

Work on laptops, then you need cloud computing and other techs.  Go to AMT, 6 months later, Docker changes its version, things break, you want to call your grad students but now they're working on their active learning startups.

NEXT tries to fix this issue. We tried to build a general system. 

Who is NEXT aimed at? The whole spectrum:
\begin{itemize}
\item
ML researchers (ex. AFRL)
\item
Experimentalists (UW Psychology)
\item
Practitioners (New Yorker, crowd-source weekly cartoon caption contest)
\end{itemize}•
What Active Learning applications can NEXT support?
For each application (classification, ranking, best arm) there are many algorithms. Have one main learning application and plug-and-play algorithms into that application.

Out of the box:
\begin{itemize}
\item
contextual bandits
\item
dueling bandits
\item 
cardinal bandits
\item
basic active classification
\item
ordinal embedding
\item
active ranking
\item
OFUL linear bandits.
\end{itemize}
It can be used for anything: clustering, metric learning...

The only parts you want to implement are: what you need to start the algorithm (initExp), how to get a query (getQuery), how you process the answer around query (processAnswer), in the background you constantly update the model (updateModel). You have a way to get the model (getModel). 

Triplets task: \url{nextml.org/simons} (Is the emotion in $i$ more like $j$ or $k$?) %Circular model?

New Yorker caption contest: \url{nextml.org/captioncontest}

How you choose to process the queries and order them is up to you.
%low latency high throughput, no drift over time

Next provides tools to track. 

``Strange fruit'' The difference between any 2 shapes is at the limit of human perception. Can we recover the 1-dimensional manifold these shapes are lying on? We used active learning algorithms and other algorithms.
Did active outperform random? We asked generalization error.

Blake Mason and Martina Rau did a similar experiment with chemistry molecules.

NLP annotation: ROCKET, Devin Conathan. 

Personalized image search: Rob Nowak et al. Zappos 50k dataset. Given a red boot, ask ``is this similar to the red boot''? Active algorithms perform significantly better. %than %Caffe features.

You can create your own UI.

We don't create sessions to track who you are now, but it wouldn't be difficult to do that.
%eye, mouse tracking

