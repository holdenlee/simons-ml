\section{Interactive Learning of Classifiers and Other Structures}

\subsection{What is interactive learning?}

Labeled data is expensive. Interactive learning hopes to reduce the amount of labeling necessary. Some examples:
\begin{enumerate}
\item
Active learning: machine queries a few labels adaptively.
\item
Explanation-based learning. In addition to lables the human gives an explanation in the form of relevant features.

Ex. highlight words that are predictive of the label in document classification.

ex. click on a part of the picture that explains the difference between the prediction and the actual label.

what is the benefit these interactions give in addition to the labels?
there's inherent ambiguity in the feedback
\item
Interactions for unsupervised learning.

For high dimensional datasets there are so many ways you can cluster it. There's no way an unsupervised algorithm knows which clustering you want.

Show a random snapshot of the clustering of a few points; a human gives feedback, ex. which 2 points should be together that aren't.

Machine has a clustering $C$ of data $X$ and wants feedback. Show human the restriction of $C$ to $O(1)$ points from $X$.
\item
Teaching: a human chooses maximally informative examples. 

In the other settings, a machine has to choose the informative examples.
\end{enumerate}•


Questions: 
\begin{enumerate}
\item
Efficient interaction algorithms: how much interaction is needed to learn?
\item
Interaction versus computational complexity: situations where interaction circumvents computational hardness.
\item
Modes of interaction: what kinds of interactions are easy and pleasant for human and provide reliable feedback?> Does it help to have a don't know option?
\item
Communication gap between human and machine: How to bridge this?
\end{enumerate}•


\subsection{Query learning of classifiers}

Typical heuristics for active learning:
\begin{itemize}
\item
Start with a pool of unlabeled data.
\item
Pick a few points at random and get their labels.
\item
Repeat: fix a classifier to the labels so far. Query the unlabelled point that is closest to the boundary, or most uncertain, or most likely to decrease overall uncertainty.
\end{itemize}•
This seems sensible, how to analyze?
The statistical learning model: there is an unknown, underlying distribution $\Pj$ on the (data, label) space, a hypothesis class $H$ of candidate classifiers, and the target is $h^*$ making fewest errors. Choose $h_n$ after seeing $n$ examples. We'd like $h_n\to h^*$ as rapidly as possible.

%Pick a few random 
You get a distribution which looks less and less like the underlying distribution though: there is underlying bias going on.

Consider the data lying on a line. We're looking for a threshold classifier. 
%agnostic

%R45, G5, RG5, G45.
Even with infinitely many labels you could converge to the wrong thing, a classifier with greater error than the best achievable, which is not consistent.

The main problem is biased sampling: how can you ensure consistence? We have to be careful of misplaced confidence. We have to be aware of confidence, do some exploration. 

Is there a generic fix to uncertainty-based heuristics?

What kind of benefits do we expect to get over random sampling?

Suppose the ground truth classifier is a threshold functions on the real line. 
In supervised learning, for error $\le \ep$, we need $\approx \rc \ep$ labeled points.
In unsupervised learning, binary search just needs $\lg\prc{\ep}$ labels, giving an exponential improvement in label complexity. What about other hypothesis classes?

For supervised learning of a hypothesis class of VC dimension $d$, we need about $\fc{d}{\ep}$ labeled points. Then there are $\le \pf{d}{\ep}^d$ ways to classify these using $H$ (Sauer's lemma). If we ask queries that cut this space in half each time, then just $d\ln \pf{d}{\ep}$ are needed. This is the dream situation. 
%grows log with size of class
%
\footnote{In teaching, you just need 2 examples: choose examples that are just to the left and right of the threshold.}

%Problems:
%\begin{enumerate}
%\item
But halving queries might not exist: there are examples where each query could be highly biased. We can pick whatever comes closest (greedy approach). Many variants have been investigated. Query by committee: do something similar with Bayesian prior.
%\end{enumerate}•

Three types of active learning:
\begin{enumerate}
\item
Mellow active learning
\item
Margin-based active learning
\item

\end{enumerate}•
\subsubsection{Mellow active learning}
Cohn, Atlas, Ladner 90's.

Separable data streams in. 
\begin{alg}[Mellow active learner]
Let $H_1$ be the hypothesis class.
Repeat for $t=1,2,\ldots$, 
\begin{itemize}
\item
receive unlabeled $x_t\in X$
\item
If there is \emph{any} disagreement within $H$ about $x_t$'s lable query label $y_t$ and set $H_{t+1}=\set{h\in H_t}{h(x_t)=y_t}$.
Else $H_{t+1}=H_t$.
\end{itemize}
\end{alg}
Only ask for labels for points in region of disagreement.
This is the most conservative learner.


There is no need to explicitly maintain $H$.

This ends up with a label for everything: either the machine asked for the label, or was 100$\%$ sure and labeled it itself. There is no sampling bias.

The label complexity can be upper-bounded in terms of the VC dimension $d$ of $H$ and the disagreement coefficient $\te$ which depends on $H$ and on the distribution $\Pj$. 
To achieve misclassification error $\ep$ with constant probability, suffices to have number of labels
$$
O\pa{\te d\ln \pf{d}{\ep}}.
$$

The intuition: Let $\Pj$ be the underlying distribution on the input space $X$. 

After $t$ points are seen, $H_t$ consists of classifiers with error at most $\approx \De_t:=\fc dt$. Let $DIS(H_t)\subeq  X$ be the part of the input space on which there is disagreement within $H_t$ Any point outside $DIS(H_t)$ is not queried.

The disagreement coefficient $\te$ satisfies $\Pj(DIS(H_t))\le \te \De_t$.

%dependence on \ep? Rechoosing ep?
The expected number of queries is
$$
\sumo tT \Pj (DIS(H_t))\le \te\sumo tT = \te \sumo tT\fc dt\le \De_t \approx \te d \ln T.
$$
Defining the disagreement coefficient requires us to get into the geometry of the hypothesis class.
The induced pseudo-metric on hypotheses is 
$$
d(h,h') = \Pj[h(X)\ne h'(X)].
$$
The ball is $B(h,r)=\set{h'\in H}{d(h,h')<\ep}$.
%For example, for $X=\R$, $H$ the set of thresholds, $B(h^*,r)$ consiste of 

%points at which disagreement.

%We only need to consider $V
The disagreement region of anyu set of candidate hyp $V\subeq H$ is $DIS(V) = \set{x\in X}{\exists h,h'\in V \text{ s.t. }h(x)\ne h'(x)}$. Need only consider $V=B(h^*,r)$, $h^*$ target hypothesis,
$$
\te = \sup_r\fc{\Pj[DIS(B(h^*,r))]}{r}.
$$
For thresholds for $\R$, $\te =2$.

For linear separators, $\te\le \sqrt d$.
%look impressive, but in practice not impressive enough.
\subsubsection{Margin-based active learning}

(Balcan-Long)

Consider algorithms based on actual heuristics, like  querying points lying close to the boundary. Query points that are within an $\ep$ margin of the boundary, and reduce $\ep$ over time.

\begin{alg}[Margin-based active learning]
For example, say all $\ve{x}=1$, $t=1,2,\ldots$, let $w_t$ be the classifier based on data so far. Randomly choose points with $|x\cdot w_t|\le m_t$, and query their labels. $\{m_t\}$ is a schedule of margins that decreases to 0.
\end{alg}
%need just d, 1/2 to 1/4, dish

Later on, you need more examples because most of the examples are useless. Never get to the point where you need $\fc d{\ep}$ examples to halve the error. Cut out the regions you're sure about. Operate in the region where the error rate is constant.



\subsubsection{Active annotation}

The schemes we covered before are fine-tuned to the classifier. %data not looking linear to me. quadratic 
If you decide to change the classifier, you have to get a different set of examples each time.

Input:
\begin{itemize}
\item
finite set of data points $\{x_1,\ldots, x_n\}$ each of which has an associated label $y_i$ that is initially missing.
\item
parameters $0<\de,\ep<1$.
\item
access to oracle that can supply any label $y_i$.
\end{itemize}

Output $\wh y_i$ such that w.p. $\ge1-\de$,
at most $\ep$ fraction of these labels are incorrect
$$\sum_i \one(y_i\ne \wh y_i)\le \ep n.
$$
The goal is to minimize calls to the oracle.

%nonparam, no assumptions

For example, the input is a neighborhood graph $G$ whose nodes are the data points $x$. Each node has an unknown label. The goal is to find the cut-edges in this graph that separate two labels.

%scheme scales with that number. 
%binary search along path. log # queries.
%number of cut edges times log diam

\begin{alg}[$S^2$ algorithm, Dasarthy-Nowak-Zhu]
Keep going until budget runs out.
\begin{itemize}
\item
If there exist labeled nodes of opposite polarity that are connected in $G$, find the shortest path connecting nodes of opposite labels, and query its midpoint. Else Choose a random point and query.
\item
Remove any newly -revealed cut edges from the graph $G$.
%Query its midpoint.
\end{itemize}
%battleship
\end{alg}
%suppose 1/100 +'s. Allowed to screw up $\ep$.
%random search that can't be overcome.

%raw unlabeled -> human label -> ML -> adaptive alg -> human labeling. ML -> predictive model

Reduce amount of effort required by human to train a machine. Also applies to minimize experimental budget.

Often adaptive sampling leads to complicating complexity, including scalability, computation, fragility to modeling assumptions. %computational burden of active learning, implementation challenges, fragility to noise, mis-modeling (Theory can be over-conservative.) Overly consrvative adaptation limits benefits over passive learning.
This is why we haven't seen active learning used so much in practice.

Does interactivity always help? Not always, but it has promise in many applications. 

Solution: identify specific ML models and apps that have demonstrated real-world successes. We go back to multi-armed bandits.

We focus on theoretical foundations of basic and linear bandit algorithms with human-machine interaction applications.

Why? The analysis and bounds are very tight (even in terms of constant factors), so we have theoretically-sound algorithms that are as aggressive as possible.

In the stochastic bandit problem, there are $n$ arms, the reward distributions $p_i$ are unknown. Each step select arm $i_t\in [n]$ and draw $x_{i_t,t}\sim p_{i_t}$ independently from past.

Examples: arms are ads, action is display an add, reward is clicks.
%13000 fruit fly
%strain of cell is arm 
%virus replif
%infect, wait.
Help scientists adapt select exper to det which genes involved in disease. Arms are genes/proteins. %, infext

Let $\mu_1>\cdots \ge \mu_n$ be the expected rewards of each arm, $\De_i = \mu_1-\mu_i$ the ``gaps'', $x_{it}\sim p_i$ indepedent random reward from arm $i$ at tie $t$.
$$
\wh \mu_{i,t} = \rc{t_i}\sumo j{t_i}x_{ij}
$$
Assume bounded, subgaussian rewards. Then for all $i$ and all $t$, we get a confidence bound on $\mu_i$.
%$$
%\mu_i \le \wh \mu_i
%$$

The regret is 
$$
R_T = T\max_i\mu_i - \E\sumo tT x_{i_t, t} = \sumo in \De_i \E T_i.
$$
%max number of clocks, total virus repliaction.d\
Sample arms to minimize $\Pj(\mu_{\wh i} \ne \max_i\mu_i)$. %identify arm that maximizes expected cclicks or virus replicatoi

%sample arm with largest upper confidence bound. This tends to equalize the 
%manages E and E together.

Eventually stop sampling suboptimal arms as long as there is some gap between the top and second best action.

Regret is $R_T = O(\sum_{i\ge 2} \fc{\ln T}{\De_i})$. Select top few. %

%are in $T=O(
For any $i$ and all $t$ w.p. $\ge 1-\de$,
$$ \wh \mu_{i,t} - 2\sqrt{\fc{\ln \ln \pf{t}{\de}}t}\le \mu_i \le \wh \mu_{i,t} + 2\sqrt{\fc{\ln \ln \pf{t}{\de}}t}.$$
%eventually we stop sampling suboptimal arms.
%UFCS samples
%sharper than Hoeffding in dependence on $t$.
%try to control confidence interval for top arm, random stopping time based on number of times...
%largest arms accidentally looks lower and you stayed there forever.

%successive elimination

This is an adaptive algorithm. How does it compare to the best non-adaptive algorithm?

We have to sample every arm as many times as the second arm, $\sum_i T_i = \wt O(n\De_2^{-2})$. We could be a factor of $n$ off.

Every week $n\approx 5000$ captions are submitted to the New Yorker. Crowdsource contest to volunteers who rate captions. The goal is to identify funniest caption. 

Over time, focus on the better captions. Little gap between theory and practice. There is a 5x improvement in sample size.

%linear, combo, Lipschitz, contextual.
%Structured bandits: the mapping from 
There is a feature vector associated with each arm $x_i\in \R^d$. The reward model is $y_i = \an{x_i, \te^*}+\ep_t$. %The optimal arm is $x^*=\amax_{\te\in C_t} \an{x_i, \te}.$
Try to construct a confidence set so $\te^*$ is always in that confidence set. Form the least-squares/ridge regression estimate of $\te$.
$$
\wh \te_t \approx \te^* + (x_t^TX_t+\la I)^{-1} X_t^T\ep_t
$$
Using martingale techniques get a confidence ellipsoid.

To select the arm to sample
$$
x_{i_t} = \amax_{x_{1:n}} \an{x_i,\wh \te_t} + \sqrt{\be_t}\sqrt{x_t^TVx_t}...
$$
The first term is the exploit term; the second is the explore term.

Interactive image search problem: User gives feedback on images from the  Zappos50K datset (shoes). From deep learning, shoes are represented by $\R^{1000}$ feature vectors.  Linear bandits retrieve 2-3 times as many relevant items.

Now we talk about systems and apps. It's difficult to even do an experiment!
There are many practical challenges. 
Researchers are not easily able to test, theoreticians may be unaware of real-world challenges with interesting math solutions; practitioners are not able to apply interactive learning methods.

Two groups working on this are Microsoft's Multiworld Testing Decision Service, and NEXT at UW-Madison.

%Alex Agarwal

%KL bounds. minimal assumptions on noise.