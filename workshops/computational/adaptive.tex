\section{Understanding Generalization in Adaptive Data Analysis (Vitaly Feldman, IBM Almaden)}

I will describe recent work on algorithms for ensuring generalization when random samples are reused to perform multiple analyses adaptively. I will also discuss connections to the problem of understanding generalization of algorithms for stochastic convex optimization and some challenging open problems.

%Models of learning inspired by nature.

(with Dwork, Hardt, Pitassi, Reingold, Roth 14, 15) New results (F, Steinke 17)

We have a learning problem described by distribution $P$ over domain $X$. Input is dataset $S=(x_1,\ldots, x_n)\sim P^n$. Do analysis $A$, get model $f=A(S)$. 
What is $\EE_{x\sim P}[loss(f,x)]$.

%Get generalization guarantees for $f$.
In practice data analysis is adaptive, depending on previous analyes of same dataset.
Starting from the second analysis, the iid assumption no longer holds. %it's dependent on data.
If we had unlimited data, we could do conservative data splitting, split into $k$ chunks, run each on separate chunk.
In practice, data is never unlimited; can we do better than this conservative approach?

``Thou shalt not test hypotheses suggested by data.''
With exception from some limited cases, there are no good solutions. It's often ignored, causing spurious conclusions. ``Quiet scandal of statistics.'' (Leo Breiman, 1992)

ML then was young. Split into training and testing. Training used to run learning algorithm, testing used to measure expected loss. If test set used only once, test error of $f$ is $\approx \E_{x\sim P}[loss(f,x)]$. But test set was used multiple times.

Since then the area has grown. Split 3-ways, training, validation, and testing.
This mitigates overfitting to the test-set but is still unsatisfying. It doesn't solve all the problem; with a lot of tuning we could overfit to validation set. The problems which are solved are solved by the same conservative data-splitting.

Find a way to mitigate dependencies. Goal: given $S$ compute $v_i$'s close to running $A_i$ on fresh samples.
Each analysis is a query. Design algorithm for answering adaptively-chosen queries. 

Queries are
$$
A_i(S) = \rc n \sum_{x\in S} \phi_i(x), \quad \phi_i:X\to [0,1].
$$
Example: $\phi_i=Loss(f,x)$. $|v_i-\E_{x\sim P}[\phi_i(x)]|\le \tau$ w.p. $1-\be$.

Can use to measure correlations, moments, accuracy/loss.
%kearns 93

Given $k$ non-adaptive query functions $\phi_1,\ldots, \phi_k$ and $n$ iid samples from $P$ estimate $\EE_{x\sim P}[\phi_i(x)]$. Use empirical mean $\E_S[\phi_i]=\rc n \sum_{x\in S}\phi_i(x)$. 

What if we use $\E_S[\phi_i]$? There are sequences of queries where the number of samples required is $n\ge \fc{k}{\tau^2}$, the kind that would often arise in boosting, etc. This is not much better than data splitting $n=O\pf{k\ln k}{t^2}$.

There exists an algorithm that can answer $k$ adaptively chosen SQ's with accuracy $\tau$ for 
$$
n=\wt O\pf{\sqrt k}{\tau^{2.5}}
$$
to estimate within $\tau$.
Data splitting gives $\wt O\pf{k}{\tau^2}$. 

BNSSSU analyze better to get $n=\wt O\pf{\sqrt k}{\tau^2}$, and generalize to low-sensitivity analyses $|A_i(S)-A_i(S')|\le \rc n$  when $S,S'$ differ in single element.

Analyses rely on differential privacy. If you look at output distribution on adjacent datasets, they will be multiplicative close except $\de$, $\Pj_M[M(S)\in Z] \le e^\ep \Pj_{M}[M(S')\in Z]+\de$.

Differential privacy implies generalization.

Differential privacy is stability. It implies strongly uniform replace-one stability and generalization in expectation.

DP composes adaptively: composition of $k$ $\ep$-DP algorithms: for every $\de>0$, is $(\ep\sqrt{k\ln \prc{\de}}, \de)$-DP. 

Take any DP algorithm, it will work in this adaptive setting. Answering low-sensitivity queries ($\max_{S,S'}|A(S)-A(S')|\le \rc n$) is the most basic algorithm in DP. Answer query $A$ with $A(S)+\ze$, ex. $\ze\sim N(0,\tau)$. 

Can we go beyond low-sensitivity queries? %It is maximum difference betwee
Error scales with worst-case sensitivity, $\De \sqrt n k^{\rc 4}$ where $\De$ is worst-case sensitivity, $\max_{A,S,S'} A(S)-A(S')$.

We show there exists an algorithm that for any adaptively chosen sequence $A_1,\ldots, A_k:X^t\to \R$ given $n=\wt O(\sqrt k t)$ iid samples from $P$ outputs $v_1,\ldots, v_k$ such that whp for all $i$,
$$
|\E_{S\sim P^t}[A_i(S)] - v_i|\le 2\si_i
$$
where $\si_i=\sqrt{\Var_{S\sim P^t} [A_i(S)]}$. 

Get error that scales with $\fc{\Var_{S\sim P^t} [A_i(S)]}{\sqrt n}k^{\rc 4}$.

Stable median algorithm: split into chunks $m$ of size $t$, $n=tm$. Find an approximate median with DP relative to $U$, value greater than bottom $\rc 3$ and smaller than top $\rc 3$ in $U$.

Use a simpler algorithm based on exponential mechanism: output $v\in T$ with probability $\propto e^{-\ep \ab{|\set{u\le u}{u\in Y}| - \fc m2}}$. Uses $O\pf{\ln r}{\ep}$. Get stability and confidence amplification for price of one log factor.

Key result: differential privacy approximately preserves quantiles. 

Let $M$ be DP algorithm that on input $U\in Y^m$ outputs $\phi:Y\to \R$, $v\in \R$. Then whp over $U\sim D^m$, $(\phi,v)=M(U)$, 
$$
\Pj_{y\sim D}[v\le \phi(y)] \approx \Pj_{y\im U}[v\le \phi(y)].
$$
If $v$ is within $\pa{\rc 3, \fc 23}$ empirical quantiles, then $v$ within $\pa{\rc 4, \fc 34}$ quantiles, mean $\pm 2\si$. 
If $\phi$ is well-concentrated on $D$ then easy to prove high-probability bounds.

Limitations: 
%\begin{itemize}
%\item
(Hardt, Ullman 14, Steinke, Ullman 15). Any algorithm for answering $k$ adaptively chosen SQ's with accuracy $\tau$ requires $n=\Om(\sqrt k/\tau)$ samples. (In high dimension or under crypto assumptions)
%\item

Verification of responses to queries: $n=O(\sqrt c \ln k)$ where $c$ is number of queries that failed verification. Get data splitting if overfitting, reusable holdout, maintain public leaderboard in competition.
%\end{itemize}

Open problems:
\begin{itemize}
\item
Queries depend only on previous answers: Does there exist a SQ analyst whose queries require more than $O(\ln k)$ samples to answer withing 0.1 accuracy/confidence?
\item
Fixed ``natural'' analyst/learning algorithm. When does adaptivity require more samples?

Ex. gradient descent for stochastic convex optimization.
\end{itemize}

Let $K=B_2^d(1)=\set{x}{\ve{x}_2\le 1}$. Let $F=\set{f\text{ convex}}{\forall x\in K\ve{\nb f(x)}_2\le 1}$ class of convex 1-Lipschitz functions. 

We don't know how many samples we need for this problem to know that the solution generalizes. Sample machinery of uniform convergence gives that $O\pf{d}{\ep^2}$ examples suffices and is necessary (F16).
But SGD solves using $O\prc{\ep^2}$ samples. 

Empirical gradient is $\nb f_S(x_t) = \rc n\sum \nb f_i(x_t)$. On fresh samples, $\ve{\nb f_S(x_t)-\nb f_P(x_t)}_2\le \rc{\sqrt n}$.
%step 2 depends on samples

Overall, get $\fc{d}{\ep^2}$ statistical queries with accuracy $\ep$ in $\rc{\ep^2}$ adaptive rounds.
How many samples do you need to answer these adaptive queries? Sample splitting gives $O\pf{\ln d}{\ep^4}$, DP gives $\wt O\pf{\sqrt d}{\ep^3}$, both incomparable and incomparable to uniform convergence bounds.

(Q: Address conjecture that SGD is form of regularizer? The issue here is reuse.)

Uniform convergence is not needed?

Further directions:

\begin{itemize}
\item
Going beyond tools from DP: other notions of stability for outcomes, max/mutual information.
\item
Generalization beyond uniform convergence. 
\item
Using in practice.
\item
Are some types of adaptivity more benign than others? %amount of dependence.  No general way to do it.
\end{itemize}•
