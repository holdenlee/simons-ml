\section{The Polytope Learning Problem (Eric Xing, Carnegie Mellon University)}

I will talk about the polytope learning problem: Given random samples from a polytope in the n-dimensional space with poly(n) vertices, can we efficiently find an approximation to the polytope? (Information theoretically, poly(n) many samples suffice.) I will discuss some approaches and related conjectures for this problem, and also its relation to other problems in learning theory.

More general problem: learning convex bodies. Let $K\in \R^d$ be unknown convex body. Input is uniformly random samples $x_1,x_2,\ldots \in K$. Problem is to reconstruct $K$.

Output is a set $K'$ that is close to $K$, $|U_K - U_{K'}|_1\le \ep$. $K'$ is given by a $\poly(d)$ time oracle (improper learning). 

How many samples/time is needed?

PAC-learning of convex bodies is hard because VC-dimension of convex bodies is unbounded. 

We are only given positive samples.

In constant dimensions it can be done efficiently using tight-fitting polytopes, required time scales as $\prc{\ep}^{\Om(d)}$. High dimensions have too much room. 

Interesting things can happen in high-dimensions with randomized algorithms With membership queries, volume computation in $\poly(d)$ time, Dyer-Frieze-Kannan 1989.

How about learning convex bodies with membership oracle?

%random points.
%why can't you output oracle as answer?
%poly many queries

$2^{\Om\pa{\sfc d\ep}}$ samples are needed to learn convex bodies in $\R^d$ from random samples and membership queries. 

Idea: Construct a large family of convex bodies in $\R^d$ such that any two bodies are far from each other. Polynomially many random samples and membership queries from any such body insufficient to determine which one it is. 

Hard family construction of KOS08: take a sphere and take intersections of exponentially many half-spaces tangent to it.

%In our case distribution adapts itself to the body.
Another lower-bound construction (GR09). Start with cross-polytope in $\R^d$, $\conv(\{\pm e_i\})$.  This has $2^d$ facets and $2d$ vertices. To each  facet attach pyramid of fixed height. 
TO know whether a pyramid is attached to facet, we must see point in that pyramid. 
But total volume of pyramid is $O\prc d$ of whole body.

Easy fix: take products of $\sqrt d$ cross-polytopes in $\R^{\sqrt d}$ with randomly attached pyramids. 
This satisfies the 2 conditions.

Lov\'asz: Can volume be estimated from poly many random samples? 

Eldan09: Exponentially many samples are needed to approximate volume. (Only samples, no oracle.)

Comvex bodies in all lower bounds have complex boundaries: exponentially many facets

Restrict convex bodies: polytopes with $m$ facets (intersection of $m$ halfspaces). VC-dimension is $O(md\ln m)$. PAC learning: $\poly_{\de,\ep}(md)$ samples suffice. Don't know how to efficiently PAC-learn intersetions of 2 half-spaces. There are many computational hardness results. 

For fixed distribution: Gaussian: $d^{\fc{\ln m}{\ep^2}}\wedge d\pf{m}{\ep}^m$ time complexity.

We need extra restriction that intersectino is bounded $m\ge d+1$. 

Restate problem: given $P\sub \R^d$ with at most $m$ facets, $\poly(m,d)$ uniformly random samples from $P$, efficiently reconstruct $P$.

If allow membership oracle, can PAC-learn  in poly time.


Approaches
\begin{enumerate}
\item
Polytope learning problem has sample complexity $\poly_{\de, \ep}(md)$. 

Min volume polytope $Q$ with at most $m$ facets and enclosing samples is good reconstruction (folklore).

For $P_1,P_2$ with at most $m$ facets, VC-dimension of $P_1\bs P_2$ is $O(dm\ln m)$. By VC uniform convergence theorem, $\rc 2|U_P-U_Q|_1= \fc{|P\bs Q|}{|P|}$ is small. 

It is NP-hard to find smallest volume enclosing simplex. But we get a structured distribution; we can use various heuristics but haven't succeeded. We don't even know how to learn cubes using this approach, but they can be learned using another.
\item
Method of moments.

Subproblem: Given samples from a parallelopiped $X=AS+b$, $S\in_U [-1,1]^d$ uniform, $A\in \R^{d\times d}$ nonsingular, $b\in \R^d$, reconstruct parallelopiped.

This reduces to learning cubes. Assume $A$ is rotation matrix and $b=0$.

The class of cubes can be learned efficiently and properly (Delfosse-Loubaton95, Frieze-Jerrun-Kannan96). The fourth directional moment of the cube is 
$$F(v) = \E_X (v\cdot X)^4.$$
If input cube is axis-aligned with unit sides, $F(v) = c_1-c_2(v_1^4+\cdots + c_n^4)$.
Local min for $\ve{v}_2=1$ are exactly facets.
%For rotation matrix $
Use projected gradient descent to find facets. 
\end{enumerate}

Learning simplices: 
FJK96 asked about learning polytopes, in particular simplices. Simplices can be learned efficiently (Anderson-Goyal-Rademacher13). Fastest algorithm for learning intersections of $d+1$ half-spaces under Gaussian distribution required $d^{\ln d}$ time. 
%NN is sum gate.

Two proofs for learning simplices:
\begin{itemize}
\item
Third directional moment: more involved than cubes
\item
Reduce to independent component analysis.
\end{itemize}

$X=AS$ where $X,S\in \R^d$ are random variables, $A$ is fixed unknown nonsingular $A\in \R^{d\times d}$. 
$(S_1,\ldots, S_d)$ are independent non-Gaussian components with unknown distribution. 
FJK, DL results apply to ICA.

Standard $d-1$ dimensional simplex is 
$$
\De_{n-1} = \set{x\in \R^d}{\sum_i x_i=1, x_i\ge 0}.
$$
If $X$ is uniform on $\De_{n-1}$, $r\ge0$ has density $\propto x^{d-1}e^{-x}$, then $X$ has identical independent nonnegative components with density $\propto e^{-x}$.

Tensor decomposition methods are a different but related way to utilize moment info, recently popular for latent variable models. 

Learning simplices is a special case of LDA, learnable by tensor decomposition. 

Can we use method of moments to learn polytopes? We can learn cubes, simplices, cross-polytopes. For general polytopes symmetry is lost. It is hard to extract facet information from moments without symmetry. 

Exact reconstruction for polytopes with $m$ vertices if given first $O(md)$ moments in $d+1$ random directions (Gravin-Lasserre,-Pasechnik-Robins12). Doesn't help with polytope learning: estimating $k$th moment seems to require $2^{\Om(k)}$ samples.
Can we learn with moments of constant order?

Estimate third directional moment in direction $v\in \R^d$ is $\E[(v,X)^3]=\sum_{i,j,k} v_iv_jv_k \E[X_iX_jX_k]$. 

Q: does 100th moment tensor suffice to determine polytopes with at most $10d$ facets information-theoretically? Does $M_{100}(P) = M_{100}(Q)$ imply $P=Q$? 
Answer is no. 

$d$-gons can't be determined for large $d$ by counting argument; lift by making prisms.

Degeneracy issue is common in applications of method of moments. For Gaussian mixture models (Kakade-Hsu12) require means be in general position.  If Gaussian are on line, sample requirement can be exponential in number of Gaussians (Moitra-Valiant10). 

%(These lower bounds are parameter estimation.) %evaluator not generator.

%For polytope learning, poly sample sizes always suffices by smallest volume enclosing polytope. 

Question: Moments of order 100 suffice to determine polytopes with at most $10d$ facets if normals are in general position?
\begin{enumerate}
\item[3.]
Radon transform: widely used technique in computerized tomography. 
Send X-ray in any direction you want, and check how much the signal is attentuated, get length of $L\cap K$. Radon transform of $K$ map lines to distance of intersection with $K$.

Radon inversion formula allows reconstruction of $K$ from $R_K(L)$ for all $L$. This works in higher dimensions too, with $d-1$ dimensional hyperplanes.

In $\R^3$: $R_K(P)$ is the are of intersection of $K$ with plane $P$. $(R^*R_K)(x)$ is the integral of $R_K(P)$ over all planes with $x\in P$. 
Then
$$
1_K(x)=-\rc{8\pi^2}\De (R^*R_K)(x). 
$$
(Similar formulas in higher dimensions.)j

$R_K(H)/\vol(H)$ is approximately the fraction of sample points in thin band around $H$ divided by width of band.%mult

Two hurdles: stability of inversion: sensitive to errors. Computation of inversion involves integral over unit sphere in $\R^n$ and multiple differentiation.

%no degeneracy problem
\end{enumerate}•
